\chapter{Programs}


\begin{definition}{Syntactic class of programs}
The phrase class Program of programs is defined as follows
\begin{equation*}
\program ::= \longprog
\end{equation*}
\end{definition}

\begin{clause}{Inclusion of files is implementation-dependent}
The Definition observes that sometimes the Standard ML compiler will be
reading in a file, and at other times we will be working with a REPL. In
either case, if an error is found, we should wish to avoid creating a
distinction between, say, batch programs and interactive programs, and
so the Definition tacitly regards all programs as interactive. This
leaves it to implementers to clarify the inclusion of files.  

Specifically, how including files affects updating the static and
dynamic basis --- this is implementation dependent behaviour.
\end{clause}

\begin{clause}{Parsing error reporting is implementation dependent}
How parsing errors are handled is left to the implementers (since it
depends on the kind of parser employed).
\end{clause}

\begin{clause}{Execution of programs}
In this section, the \define{Execution} of a program means the combined
elaboration and evaluation of the program.
\end{clause}

\begin{definition}{Class of Basis, StaticBasis, DynamicBasis}
The Definition wishes to disambiguate the usage of $\B$ for bases (since
it was used as the metavariable for both static and dynamic bases).
For ease of discussion, the Definition does the following:
\begin{enumerate}
\item In giving the semantics of programs, however, let us rename as 
\mbox{StaticBasis} the class  Basis defined in the static
semantics of modules~\zref{defn:static-modules:compound-objects}, Section~\ref{sec:static-modules:compound-objects},
and let us use $\Bstat$ to range over StaticBasis.
\item Let us rename as  \mbox{DynamicBasis} the class Basis defined in the
dynamic semantics of modules~\zref{defn:dynamic-modules:compound-objects}, Section~\ref{sec:dynamic-modules:compound-objects},
and let us use $\Bdyn$ to range over \mbox{DynamicBasis}.
\item We now define
  \begin{equation*}
\B\ {\rm or} \ (\Bstat,\Bdyn)\in{\rm Basis}={\rm Static Basis}
\times {\rm DynamicBasis}.
  \end{equation*}
\end{enumerate}
\end{definition}

\begin{definition}{Judgement forms}
The Definition uses $\tsstat$ for elaboration as defined in 
Section [Chapter]~\ref{ch:static-modules}, and $\tsdyn$ for evaluation
as defined in Section [Chapter]~\ref{ch:dynamic-modules}.
Then $\ts$ will be reserved for the execution of
programs, which thus is expressed by a sentence of the
form
\begin{equation*}
\s,\B\ts\program\ra\B',\s'
\end{equation*}
This may be read as follows: 
starting in basis $\B$ with state $\s$ the execution of
$\program$ results in a basis $\B'$ and a state $\s'$.
\end{definition}

\begin{clause}{Programs never result in exceptions}
Executing a program never results in an exception. If the evaluation of
a $\topdec$ yields an exception (for instance because of a $\RAISE$ expression),
then the result of executing the program ``$\topdec\ \mbox{\ml{;}}$''
is the original basis together with the state which is in force when the
exception is generated.

In particular, this means the exception convention~\zref{convention:dynamic-modules:state-and-exception}
is not applicable to the rules the Definition describes in this section
[chapter]. 
\end{clause}

\begin{convention}{Non-elaboration of top-level declaration}
The Definition represents the non-elaboration of a top-level declaration by
 ``$\ldots\tsstat\topdec\not\ra$''.
\end{convention}

\rulesec{Programs}{\s,\B\ts\program\ra\B',s'}

% 187
\begin{inference-rule}{Failing elaboration}
\begin{equation}\label{rule:programs:}
\vcenter{\infer{\s,\B\ts\longprog\ra\B\optional{'},s\optional{'}}
  {\of{\Bstat}{\B}\tsstat\topdec\not\ra
    & \optional{\s,\B\ts\program\ra\B',s'}}}
\end{equation}
A failing elaboration has no effect whatever.
\end{inference-rule}

\begin{remark}{Implications of negative premise in rule 187}
Rossberg~\cite{rossberg2018defects} observes the negative premise in
rule 187 has unfortunate implications.

If we interpret the rule strictly, it precludes any conforming
implementation from proving any sort of conservative semantic extension
to the language. Any extension that allows declarations to elaborate
that would be illegal according to the Definition (e.g., polymorphic
records) can be observed through this rule and change the behaviour of
consecutive declarations. This could alter the observed behaviour of a
program. Consider Rossberg's example:
\begin{quote}\parskip=0pt
  \texttt{val s = "no";}

  \strdec

  \texttt{val s = "yes";}

  \texttt{print s;}
\end{quote}
where the $\strdec$ only elaborates if some extension is supported. In
that case, the program will print \texttt{yes}, otherwise it will print
\texttt{no}.

Rossberg interprets this as suggesting that formalizing an interactive
toplevel is not worth the trouble. He's probably right.
\end{remark}

% 188
\begin{inference-rule}{Evaluation raising exception}
\begin{equation}\label{rule:programs:evaluation-raising-exception}
\vcenter{\infer{\s,\B\ts\longprog\ra\B\optional{'},\s'\optional{'}}
  {\begin{array}{lr}
      \of{\Bstat}{\B}\tsstat\topdec\ra\Bstat^{(1)} & \\
      \s,\of{\Bdyn}{\B}\tsdyn\topdec\ra\p,\s' &
                  \optional{\s',\B\ts\program\ra\B',\s''}
   \end{array}}}
\end{equation}
An evaluation which yields an exception nullifies the change in the
static basis, but does not nullify side-effects on the state which may
have occurred before the exception was raised.
\end{inference-rule}

% 189
\begin{inference-rule}{Success}
\begin{equation}\label{rule:programs:success}
\vcenter{\infer{\s,\B\ts\longprog\ra\B'\optional{'},\s'\optional{'}}
  {\begin{array}{ll}
    \of{\Bstat}{\B}\tsstat\topdec\ra\Bstat^{(1)} & \\
    \s,\of{\Bdyn}{\B}\tsdyn\topdec\ra\Bdyn^{(1)},s'&
                \B'=\B\oplus(\Bstat^{(1)},\Bdyn^{(1)})\\
    &           \optional{\s',\B'\ts\program\ra\B'',s''}
   \end{array}}}
\end{equation}
\end{inference-rule}

\begin{remark}{Ambiguity in semicolons}
Recall~\zref{clause:syntax-modules:top-declaration-restriction} ``No \nonterminal{topdec} may contain, as an initial segment, a
\nonterminal{strdec} followed by a semicolon.'' Kahr~\cite[\oldS8.3]{kahrs1993mistakes}
and Rossberg~\cite{rossberg2018defects} both point out this is a problem.

Kahr points out that ``$\dec_{1}\texttt{;}\dec_{2}\texttt{;}\dec_{3}$''
can be parsed in two different ways (at least), taking $\dec_{12}$ as $\dec_{1}\texttt{;}\dec_{2}$
and $\dec_{23}$ as $\dec_{2}\texttt{;}\dec_{3}$, these evaluate to the
same thing provided in rule~\ref{rule:static-core:sequential-declaration}
and rule~\ref{rule:dynamic-core:sequential-declaration} we can have
\begin{equation*}
\C\oplus(\E_{1}\oplus\E_{2})=(\C\oplus\E_{1})\oplus\E_{2}.
\end{equation*}
Unfortunately, this does not hold, and the right-hand side of this
equation can contain more type names than the one of the left-hand side.
This means that the associativity of $\optional{\texttt{;}}$ is not obvious.

Rossberg resolves this problem by having HaMLet's parser reduce to
$\strdec$ as soon as possible (since this solves other problems as
well). This seems to be the ``spirit of the Law'', even if the ``letter
of the Law'' leaves much to be desired.
\end{remark}

\begin{definition}{Core language programs}
A program is called a \define{Core Language Program} if it can be parsed
in the reduced grammar defined as follows:
\begin{enumerate}
\item Replace the definition of top-level declarations by
\begin{equation*}
\topdec\ ::=\ \strdec
\end{equation*}
\item Replace the definition of structure-level declarations by
\begin{equation*}
\strdec\ ::=\ \dec
\end{equation*}
\end{enumerate}
\end{definition}
