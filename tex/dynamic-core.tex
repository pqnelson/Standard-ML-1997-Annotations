\chapter{Dynamic Semantics for the Core}

\section{Reduced Syntax}

\begin{clause}{Core syntax reduced}
Since types are mostly dealt with in the static semantics, the Core
syntax is reduced by the following transformations, for the purpose of
the dynamic semantics:
\begin{enumerate}
\item All explicit type ascriptions ``\ml{:} $\ty$\,'' are omitted, and
      qualifications ``$\OF\  \ty\,$'' are omitted from constructor and
      exception bindings.
\item The Core phrase classes Ty and TyRow are omitted.
\end{enumerate}
\end{clause}

\section{Simple Objects}

\begin{definition}{Classes of simple objects}
All objects in the dynamic semantics are built from the identifier
classes together with the simple object classes listed below (note that
basic values are discussed in their own section):
\begin{equation*}
\begin{array}{rclr}
\A               & \in   & \Addr	& \mbox{addresses}\\
\e               & \in   & \Exc 	& \mbox{exception names}\\
b      		& \in	& \BasVal	& \mbox{basic values}\\
\sv             & \in   & \SVal         & \mbox{special values}\\
                &       & \{\FAIL\}     & \mbox{failure}
\end{array}
\end{equation*}
\end{definition}

\begin{clause}{Infinite set of addresses}
\Addr\ forms an infinite set.
\end{clause}

\begin{clause}{Infinite set of exception names}
\Exc\ forms an infinite set.
\end{clause}

\begin{clause}{Special values are values of special constants}
{\SVal} is the class of values denoted by the special constants
\SCon. Each integer, word, or real constant denotes a value according to normal 
mathematical conventions; each string or character
constant denotes a sequence of characters as explained in Section 2.2.
\end{clause}

\begin{definition}{Value of a special constant}\index{$\sconval(\scon)$}
The value denoted by {\scon} is written $\sconval(\scon)$.
\end{definition}

\begin{clause}{Failure ``value''}
$\FAIL$ is the result of a failing attempt to match a value and a
pattern. Thus $\FAIL$ is neither a value nor an exception, but simply
a semantic object used in the rules to express operationally
how matching proceeds.
\end{clause}

\begin{clause}{Exception constructors evaluate to names}
Exception constructors evaluate to exception names. This accomodates the
generative nature of exception bindings: each evaluation of a
declaration of an exception constructor binds it to a new unique name.
\end{clause}

\section{Compound Objects}

\begin{definition}{Compound objects}
The compound objects for dynamic semantics are listed thus:
\begin{equation*}
\begin{array}{rcl}
        \V	&\in	&\Val =\{\mbox{\tt :=}\}\cup\SVal\cup\BasVal\cup\VId\\
                &       &\qquad\quad\cup\:(\VId\times\Val)\cup\ExVal\\
                &       &\qquad\quad\cup\:\Record\cup\Addr\cup\FcnClosure\\
        \r      & \in   & \Record =  \finfun{\Lab}{\Val}\\
{\exval}      & \in   & \ExVal = \Exc \cup (\Exc\times\Val)\\
{[\exval]}\ {\rm or}\ \p
                & \in   & \Pack = \ExVal\\
(\match,\E,\VE) & \in   & \FcnClosure = \Match\times\Env\times\ValEnv\\
        \mem    & \in   & \Mem = \finfun{\Addr}{\Val}\\
        \excs   & \in   & \ExcSet = \Fin(\Exc)\\
(\mem,\excs)\ {\rm or}\ \s
                & \in   & \State = \Mem\times\ExcSet\\
(\SE,\TE,\VE\ {\rm or}\ \E
                & \in   & \Env = \StrEnv\times\TyEnv\times\ValEnv\\
        \SE     & \in   & \StrEnv = \finfun{\StrId}{\Env}\\
        \TE     & \in   & \TyEnv = \finfun{\TyCon}{\ValEnv}\\
        \VE	& \in	& \ValEnv = \finfun{\VId}{\Val\times\IdStatus}
\end{array}
\end{equation*}
\end{definition}

\begin{convention}{Projection, injection, modification}
The notions of projection, injection, and modification are adapted from
the static semantics.
\end{convention}

\begin{clause}{Object classes disjoint}
The object classes thus defined are all disjoint, so there is no
ambiguity in taking unions as disjoint unions.
\end{clause}

\begin{clause}{Exception values and packets}
Observe that $\ExVal$ and $\Pack$ (exception values and packets) are
isomorphic classes, but the latter class corresponds to exceptions which
have been raised, and therefore has different semantic significance from
the former, which is just a subclass of values.
\end{clause}

\begin{remark}{Caution about overloaded notation}
Although the same names (e.g., $\E$ for an environment) are used
as in the static semantics, the objects denoted are different.
This need cause no confusion since the static and dynamic semantics are
presented separately. 
\end{remark}

\section{Basic Values}

\begin{clause}{Simplified set of basic values}
The Definition takes \BasVal\ to be the singleton set
$\{\,\texttt{=}\,\}$, but allows for a larger set of basic values
defined by libraries.
\end{clause}

\begin{clause}{APPLY gives meaning to basic values}
The semantical meaning of basic values is represented by a function
\begin{equation*}
\APPLY\ :\ \BasVal\times\Val\to\Val\cup\Pack
\end{equation*}
which satisfies $\APPLY(\texttt{=},\{1\mapsto v_{1},2\mapsto v_{2}\})$
is \texttt{true} if $v_{1}$ and $v_{2}$ are identical values, and
\texttt{false} otherwise.
\end{clause}

\begin{remark}{Problems with APPLY}
As Rossberg~\cite[\oldS6]{rossbert2018defects} and
Kahrs~\cite[\oldS\oldS11.1--3]{kahrs1993mistakes} note, the APPLY function has no access to
program state. This implies the library primitives may not be stateful,
and moreover that a lot of interesting primitives (e.g., arrays, I/O
streams, etc.) could not be added to the language without extending the
Definition itself.

But Rossberg observes, any nontrivial library type requires an extension
of the Definition of values or state anyways (and equality types ---
for, e.g., \texttt{array}). The Definition should have contained a
comment regarding this.
\end{remark}

\section{Basic Exceptions}

\begin{clause}{Initial basis}
A subset $\BasExc\subset\Exc$ of the exception names are bound to
predefined exception constructors in the initial dynamic basis. These
names are denoted by the identifiers to which they are bound in the
initial basis, and are as follows:
\begin{equation*}
\texttt{Match\ \ Bind}
\end{equation*}
These exceptions are raised upon failure of pattern-matching in
evaluating a function or a \valbind, as detailed inthe rules that follow.
\end{clause}

\begin{clause}{Match raised by nonexhaustive patterns}
The exception \ml{Match} can only be raised for a match which is not
exhaustive, and has therefore been flagged by the compiler~\zref{clause:static-core:warn-about-nonexhaustive-patterns}.
\end{clause}

\section{Function Closure}

\begin{remark}{Closures are ``function values''}
Intuitively, a function closure is precisely the ``value'' for a
function expression. This is critical when trying to use, e.g.,
equational reasoning for Standard ML.
\end{remark}

\begin{clause}{Informal explanation of closures}
A function closure $(\match,\E,\VE)$ may be informally described as follows:
when the function closure is applied to a value $\V$, $\match$ will be
evaluated against $\V$, in the environment $\E$ modified in a special
sense by $\VE$ (to allow for recursive functions). This happens by
evaluating $\match$ in $\plusmap{\E}{\Rec\VE}$, which will be defined shortly.
\end{clause}

\begin{definition}{Rec operator on value environments}
The function
\begin{equation*}
\Rec\ :\ \ValEnv\to\ValEnv
\end{equation*}
is defined as follows:
\begin{enumerate}
\item $\Dom(\Rec\VE)=\Dom\VE $
\item If $\VE(\vid)\notin\FcnClosure\times\{\isv\}$,
      then $(\Rec\VE)(\vid)=\VE(\vid)$
\item If $\VE(\vid)=((\match',\E',\VE'), \isv)$
      then $(\Rec\VE)(\vid)=((\match',\E',\VE), \isv)$
\end{enumerate}
\end{definition}

\begin{remark}{$\Rec\VE$ ``unrolls'' function closures once}
The effect of this definition is that --- before the application of the
closure $(\match,\E,\VE)$ to $\V$ --- the function closures in $\Ran\VE$
[not a type: $\Ran\VE$ is correct, see next example]
are ``unrolled'' once, to prepare for their possible recursive
application during the evaluation of $\match$ upon $\V$.

This ensures all semantic objects are finite (by controlling the
unrolling of recursion).
\end{remark}

% TODO: finish this
\begin{example}{Example unrolling recursive function calls}
It makes more sense to see this rule in action. I'm adapting the example
from the Commentary~\cite[pp.1--12]{milner1991commentary}. Consider the
example code:
\begin{Verbatim}
datatype NAT = Zero | Succ of NAT;

fun twice(Zero) = Zero
  | twice(Succ x) = Succ(Succ(twice(x)));

twice(Zero);
\end{Verbatim}
The desugared code ``\texttt{twice = fn} $\match$'' will evaluate to
\begin{equation*}
\VE = \{\texttt{twice}\mapsto(\match,E_{1},\emptymap)\}.
\end{equation*}
The problem is that, looking at the $\VE$ appearing in the closure bound
to \texttt{twice}: it's precisely $\VE_{\mathtt{twice}}=\emptymap$! So trying to lookup
the identifier \texttt{twice} for a recursive function call will result
in a failure.

However, we don't look up \texttt{twice} in $\VE_{\mathtt{twice}}=\emptymap$
but instead we're using
\begin{equation*}
\Rec\VE = \{\texttt{twice}\mapsto(\match,E_{1},\VE)\}.
\end{equation*}
This allows for proper recursive function calls.
\end{example}

\begin{remark}{Used only twice}
The $\Rec$ operator is used only twice: in Rules (102) [the rule for
  evaluating an application expression ``$\exp$ $\atexp$''], and (126)
[governing recursive value bindings of the form ``\texttt{rec} $\valbind$''].
\end{remark}

\section{Inference Rules}

\begin{definition}{Judgement form}
The semantic rules govern judgemental forms like
\begin{equation*}
\s,A\ts\phrase\ra A',\s'
\end{equation*}
where $A$ is usually an environment, $A'$ is some semantic object, and
here $s$, $s'$ are states before and after the evaluation which occurs
in the judgement.
\end{definition}

\begin{definition}{Side-conditions}
Some hypotheses in the rules are not of the form we just described. They
are called \define{Side-Conditions}.
\end{definition}

\begin{clause}{Convention for options}
The convention for options is the same as for the Core static semantics.
\end{clause}

\begin{convention}{State convention}\label{convention:dynamic-convention:state}
In most cases, the state can be omitted from the inference rule. When
omitted, the conventions for restoring them as is follows:
If the rule is presented in the
form
\begin{equation*}
 \frac{ \begin{array}{r}
        A_{1}\ts\phrase_{1}\ra A'_{1}\qquad
        A_{2}\ts\phrase_{2}\ra A'_{2}\quad\cdots\\
        \cdots\quad A_{n}\ts\phrase_{n}\ra A'_{n}
        \end{array}
      }
      { A\ts\phrase\ra A'}
\end{equation*}
then the full form is intended to be
\begin{equation*}
\frac{ \begin{array}{r}
       \s_{0},A_{1}\ts\phrase_{1}\ra A'_{1},\s_{1}\qquad
       \s_{1},A_{2}\ts\phrase_{2}\ra A'_{2},\s_{2}\quad\cdots\\
       \cdots\quad\s_{n-1},A_{n}\ts\phrase_{n}\ra A'_{n},\s_{n}
       \end{array}
     }
     { \s_{0},A\ts\phrase\ra A',\s_{n}}
\end{equation*}
(Any side-conditions are left unaltered).

Thus the left-to-right order of the hypotheses indicates the order of
evaluation.  Note that in the case $\n=0$, when there are no hypotheses
(except possibly side-conditions), we have $\s_{n}=\s_{0}$; this implies that the
rule causes no side effect.

The Definition calls this convention the \define{State Convention}, and
it must be applied to each version of a rule obtained by inclusion or
omission of its options.
\end{convention}

\begin{convention}{Exception convention}\label{convention:dynamic-convention:exception}
The \define{Exception Convention} is adopted to deal with the
propagation of exception packets $p$. For each rule whose full form
(ignoring side-conditions) is
\begin{equation*}
\infer{ \s,A\ts\phrase\ra A',\s'}
      { \s_{1},A_{1}\ts\phrase_{1}\ra A'_{1},\s'_{1}
        & \cdots
        & \s_{n},A_{n}\ts\phrase_{n}\ra A'_{n},\s'_{n} }
\end{equation*}
and for each $k$, $1\leq k\leq n$, for which the result $A'_{k}$ is not a
packet $\p$, an extra rule is added of the form
\begin{equation*}
\infer{ \s,A\ts\phrase\ra \p',\s'}
      { \s_{1},A_{1}\ts\phrase_{1}\ra A'_{1},\s'_{1}
        & \cdots
        & \s_{k},A_{k}\ts\phrase_{k}\ra \p',\s' }
\end{equation*}
where $\p'$ does not occur in the original rule. [The only exception is
  rule 104, describing how exception handlers work.]
This indicates that evaluation of phrases in the hypothesis terminates with the
first whose result is a packet (other than one already treated in the rule),
and this packet is the result of the phrase in the conclusion.
\end{convention}

\begin{convention}{Compound variables}
The last convention allows for ``compound variables'' (built from the
metavariables and a slash ``$/$'') to range over unions of semantic
objects.

For instance 
the compound variable $\V/\p$ ranges
over $\Val\cup\Pack$. 
We also allow $x/\FAIL$ to range over $X\cup\{\FAIL\}$ where $x$ 
ranges over $X$;
furthermore, we extend environment modification to allow for failure
as follows:
\begin{equation*}
\VE+\FAIL=\FAIL.
\end{equation*}
\end{convention}

\rulesec{Atomic Expressions}{\E\ts\atexp\ra\V/\p}

% 90
\begin{inference-rule}{Special constant}
\begin{equation}\label{rule:dynamic-core:sconexp}
\vcenter{\infer{\E\ts\scon\ra\sconval(\scon)}
  {}}
\end{equation}
\end{inference-rule}

% 91
\begin{inference-rule}{Value variable}
\begin{equation}\label{rule:dynamic-core:varexp}
\vcenter{\infer{\E\ts\longvid\ra\V}
  {\E(\longvid)=(\V,\is)}}
\end{equation}
As in the static semantics, value identifiers are looked up in the
environment and the identifier status is not used.
\end{inference-rule}

% 92
\begin{inference-rule}{Record expression}
\begin{equation}\label{rule:dynamic-core:recexp}
\vcenter{\infer{\E\ts\lttbrace\ \recexp\ \rttbrace\ra\emptymap
                                  \optional{ +\ \r}\ \In\ \Val}
  {\optional{\E\ts\labexps\ra\r}}}
\end{equation}
\end{inference-rule}

% 93
\begin{inference-rule}{Let expressions}
\begin{equation}\label{rule:dynamic-core:let}
\vcenter{\infer{\E\ts\letexp\ra\V}
  {\E\ts\dec\ra\E'
    &\E+\E'\ts\exp\ra\V}}
\end{equation}
\end{inference-rule}

% 94
\begin{inference-rule}{Parenthesised expression}
\begin{equation}\label{rule:dynamic-core:parenthesised-expression}
\vcenter{\infer{\E\ts\parexp\ra\V}
  {\E\ts\exp\ra\V}}
\end{equation}
\end{inference-rule}

\rulesec{Expression Rows}{\E\ts\labexps\ra\r/\p}

% 95
\begin{inference-rule}{Labelled expressions}
\begin{equation}\label{rule:dynamic-core:labexps}
\vcenter{\infer{\E\ts\longlabexps\ra\{\lab\mapsto\V\}\optional{ +\ \r}}
    {\E\ts\exp\ra\V
      & \optional{\E\ts\labexps\ra\r}}}
\end{equation}
We may think of components as being evaluated from left to right,
because of the state and exception conventions.
\end{inference-rule}

\rulesec{Expressions}{\E\ts\exp\ra\V/\p}

% 96
\begin{inference-rule}{Atomic expressions}
\begin{equation}\label{rule:dynamic-core:atexp}
\vcenter{\infer{\E\ts\atexp\ra\V}{\E\ts\atexp\ra\V}}
\end{equation}
\end{inference-rule}

% 97
\begin{inference-rule}{Constructor application}
\begin{equation}\label{rule:dynamic-core:conapp}
\vcenter{\infer{\E\ts\appexp\ra(\vid,\V)}
  {\E\ts\exp\ra\vid
    & \vid\neq\REF
    & \E\ts\atexp\ra\V}}
\end{equation}
Note that none of the rules for function application has a premise
in which the operator evaluates to a constructed value, a record or an
address. This is because we are interested in the evaluation of
well-typed programs only, and in such programs $\exp$ will always have a
functional type.
\end{inference-rule}

% 98
\begin{inference-rule}{Exception constructor application}
\begin{equation}\label{rule:dynamic-core:exception-constructor-application}
\vcenter{\infer{\E\ts\appexp\ra(\e,\V)}
  {\E\ts\exp\ra\e
    & \E\ts\atexp\ra\V}}
\end{equation}
Note that none of the rules for function application has a premise
in which the operator evaluates to a constructed value, a record or an
address. This is because we are interested in the evaluation of
well-typed programs only, and in such programs $\exp$ will always have a
functional type.
\end{inference-rule}

% 99
\begin{inference-rule}{Reference application}
\begin{equation}\label{rule:dynamic-core:refapp}
\vcenter{\infer{\s,\E\ts\appexp\ra\A,\ \s''+\{\A\mapsto\V\} }
  {\s,\E\ts\exp\ra~\ml{ref}~,\s'
    & \s',\E\ts\atexp\ra\V,\s''
    & \A\notin\Dom(\of{\mem}{\s''})}}
\end{equation}
\begin{enumerate}
\item The side condition ensures that a new address is chosen. There are no
  rules concerning disposal of inaccessible addresses.
\item Note that none of the rules for function application has a premise
in which the operator evaluates to a constructed value, a record or an
address. This is because we are interested in the evaluation of well-
typed programs only, and in such programs $\exp$ will always have a
functional type.
\end{enumerate}
\end{inference-rule}

% 100
\begin{inference-rule}{Assignment application}
\begin{equation}\label{rule:dynamic-core:assapp}
\vcenter{\infer{\s,\E\ts\appexp\ra\emptymap\ \In\ \Val,\ \s''+\{\A\mapsto\V\}}
  {\s,\E\ts\exp\ra~\mbox{\tt :=}~,\s'
    & \s',\E\ts\atexp\ra\{{\tt 1}\mapsto\A,\ {\tt 2}\mapsto\V\},\s''}}
\end{equation}
Note that none of the rules for function application has a premise
in which the operator evaluates to a constructed value, a record or an
address. This is because we are interested in the evaluation of
well-typed programs only, and in such programs $\exp$ will always have a
functional type.
\end{inference-rule}

% 101 
\begin{inference-rule}{Basic function application}
\begin{equation}\label{rule:dynamic-core:basis-function-application}
  \vcenter{\infer{\E\ts\appexp\ra\V'/\p}
    {\E\ts\exp\ra b
      & \E\ts\atexp\ra\V
      & \APPLY(b,\V)=\V'/\p}}
\end{equation}
Note that none of the rules for function application has a premise
in which the operator evaluates to a constructed value, a record or an
address. This is because we are interested in the evaluation of
well-typed programs only, and in such programs $\exp$ will always have a
functional type.
\end{inference-rule}

% 102
\begin{inference-rule}{Closure application}
\begin{equation}\label{rule:dynamic-core:closapp}
\vcenter{\infer{\E\ts\appexp\ra\V'}
  {\begin{array}{c}
      \E\ts\exp\ra(\match,\E',\VE)\qquad\E\ts\atexp\ra\V\\
      \E'+\Rec\VE,\ \V\ts\match\ra\V'
      \end{array}}}
\end{equation}
Note that none of the rules for function application has a premise
in which the operator evaluates to a constructed value, a record or an
address. This is because we are interested in the evaluation of
well-typed programs only, and in such programs $\exp$ will always have a
functional type.
\end{inference-rule}

% 103
\begin{inference-rule}{Failing closure application}
\begin{equation}\label{rule:dynamic-core:closapp1}
\vcenter{\infer{\E\ts\appexp\ra[{\tt Match}]}
  {\begin{array}{c}
      \E\ts\exp\ra(\match,\E',\VE)\qquad\E\ts\atexp\ra\V\\
      \E'+\Rec\VE,\ \V\ts\match\ra\FAIL
   \end{array}}}
\end{equation}
Note that none of the rules for function application has a premise
in which the operator evaluates to a constructed value, a record or an
address. This is because we are interested in the evaluation of
well-typed programs only, and in such programs $\exp$ will always have a
functional type.
\end{inference-rule}

% 104
\begin{inference-rule}{Unexceptional handler}
\begin{equation}\label{rule:dynamic-core:handlexp}
\vcenter{\infer{\E\ts\handlexp\ra\V}
  {\E\ts\exp\ra\V}}
\end{equation}
This is the only rule to which the exception convention does not apply.
If the operator evaluates to a packet then rule 105 or rule 106 must be
used.
\end{inference-rule}

% 105
\begin{inference-rule}{Handled exception}
\begin{equation}\label{rule:dynamic-core:handlexp2}
\vcenter{\infer{\E\ts\handlexp\ra\V}
  {\E\ts\exp\ra[\exval]\qquad\E,\exval\ts\match\ra\V}}
\end{equation}
\end{inference-rule}

% 106
\begin{inference-rule}{Unhandled exception}
\begin{equation}\label{rule:dynamic-core:handlexp3}
\vcenter{\infer{\E\ts\handlexp\ra[\exval]}
  {\E\ts\exp\ra[\exval]
    & \E,\exval\ts\match\ra\FAIL}}
\end{equation}
Packets that are not handled by the $\match$ propagate.
\end{inference-rule}

% 107
\begin{inference-rule}{Raise exception}
\begin{equation}\label{rule:dynamic-core:raisexp}
\vcenter{\infer{\E\ts\raisexp\ra[\exval]}
  {\E\ts\exp\ra\exval}}
\end{equation}
\end{inference-rule}

% 108
\begin{inference-rule}{Function}
This evaluates a function expression to a function value.
\begin{equation}\label{rule:dynamic-core:funexp}
\vcenter{\infer{\E\ts\fnexp\ra(\match,\E,\emptymap)}
  {}}
\end{equation}
The third component of the function closure is empty because the
match does not introduce new recursively defined values.
\end{inference-rule}

\rulesec{Matches}{\E,\V\ts\match\ra\V'/\p/\FAIL}

% 109
\begin{inference-rule}{Match 1}
This intuitively formalizes ``Go with the first match possible'' semantics.
\begin{equation}\label{rule:dynamic-core:match1}
\vcenter{\infer{\E,\V\ts\longmatch\ra\V'}{\E,\V\ts\mrule\ra\V'}}
\end{equation}
\end{inference-rule}

% 110
\begin{inference-rule}{Match 2}
\begin{equation}\label{rule:dynamic-core:match2}
\vcenter{\infer{\E,\V\ts\mrule\ra\FAIL}{\E,\V\ts\mrule\ra\FAIL}}
\end{equation}
\end{inference-rule}

% 111
\begin{inference-rule}{Match 3}
Given a leading $\mrule$ which fails to match, but there is a rule in
the rest of the $\match$ which either matches or fails\dots we go with
that instead.
\begin{equation}\label{rule:dynamic-core:match3}
\vcenter{\infer{\E,\V\ts\longmatcha\ra\V'/\FAIL}
  {\E,\V\ts\mrule\ra\FAIL
    & \E,\V\ts\match\ra\V'/\FAIL}}
\end{equation}
Comment: A value $\V$ occurs on the left of the turnstile, in evaluating
a $\match$. We may think of a $\match$ as being evaluated \emph{against}
a value; similarly, we may think of a pattern as being evaluated
\emph{against} a value. Alternative match rules are tried from left to
right.
\end{inference-rule}

\rulesec{Match Rules}{\E,\V\ts\mrule\ra\V'/\p/\fail}

% 112
\begin{inference-rule}{Match a clause successfully}
\begin{equation}\label{rule:dynamic-core:mrule1}
\vcenter{\infer{\E,\V\ts\longmrule\ \ra\V'}
  {\E,\V\ts\pat\ra\VE
    & \E+\VE\ts\exp\ra\V'}}
\end{equation}
\end{inference-rule}

% 113
\begin{inference-rule}{Failing to match a pattern}
\begin{equation}\label{rule:dynamic-core:mrule2}
\vcenter{\infer{\E,\V\ts\longmrule\ \ra\FAIL}{\E,\V\ts\pat\ra\FAIL}}
\end{equation}
\end{inference-rule}

\rulesec{Declarations}{\E\ts\dec\ra\E'/\p}

% 114
\begin{inference-rule}{Value declaration}
\begin{equation}\label{rule:dynamic-core:value-declaration}
\vcenter{\infer{\E\ts\explicitvaldec\ra\VE\ \In\ \Env}
  {\E\ts\valbind\ra\VE}}
\end{equation}
\end{inference-rule}

% 115
\begin{inference-rule}{Type alias}
\begin{equation}\label{rule:dynamic-core:type-alias}
\vcenter{\infer{\E\ts\typedec\ra\TE\ \In\ \Env}
  {\ts\typbind\ra\TE}}
\end{equation}
\end{inference-rule}

% 116
\begin{inference-rule}{Datatype declaration}
\begin{equation}\label{rule:dynamic-core:datatype-declaration}
\vcenter{\infer{\E\ts\datatypedec\ra(\VE,\TE)\ \In\ \Env}
  {\ts\datbind\ra\VE,\TE}}
\end{equation}
\end{inference-rule}

% 117
\begin{inference-rule}{Datatype replication}
\begin{equation}\label{rule:dynamic-core:datatype-replication}
\vcenter{\infer{
    \E\ts\datatyperepldec\ra
    (\VE,\{\tycon\mapsto\VE\})\ \In\ \Env
  }
  {\E(\longtycon) = \VE}}
\end{equation}
\end{inference-rule}

% 118
\begin{inference-rule}{Abstract typing declaration}
\begin{equation}\label{rule:dynamic-core:abstype}
\vcenter{\infer{\E\ts\abstypedec \ra \E'}
  {\ts\datbind\ra\VE,\TE
    & \E+\VE\ts\dec\ra \E'}}
\end{equation}
\end{inference-rule}

% 119
\begin{inference-rule}{Exception declaration}
\begin{equation}\label{rule:dynamic-core:exception-declaration}
\vcenter{\infer{\E\ts\exceptiondec\ra\VE\ \In\ \Env }
  {\E\ts\exnbind\ra\VE }}
\end{equation}
\end{inference-rule}

% 120
\begin{inference-rule}{Local declaration}
\begin{equation}\label{rule:dynamic-core:local-declaration}
\vcenter{\infer{\E\ts\localdec\ra\E_{2}}
  {\E\ts\dec_{1}\ra\E_{1}
    & \E+\E_{1}\ts\dec_{2}\ra\E_{2}}}
\end{equation}
\end{inference-rule}

% 121
\begin{inference-rule}{Open declaration}
\begin{equation}\label{rule:dynamic-core:open-declaration}
\vcenter{\infer{ \E\ts\openstrdec\ra \E_{1} + \syndots + \E_{n} }
  { \E(\longstrid_{1})=\E_{1}
    &\syndots
    &\E(\longstrid_{n})=\E_{n} }}
\end{equation}
\end{inference-rule}

% 122
\begin{inference-rule}{Empty declaration}
\begin{equation}\label{rule:dynamic-core:empty-declaration}
\vcenter{\infer{\E\ts\emptydec\ra\emptymap\ \In\ \Env}
  {}}
\end{equation}
\end{inference-rule}

% 123
\begin{inference-rule}{Sequential declaration}
\begin{equation}\label{rule:dynamic-core:sequential-declaration}
\vcenter{\infer{\E\ts\seqdec\ra\plusmap{E_{1}}{E_{2}}}
  {\E\ts\dec_{1}\ra\E_{1}
    &\E+\E_{1}\ts\dec_{2}\ra\E_{2}}}
\end{equation}
\end{inference-rule}

\rulesec{Value Bindings}{\E\ts\valbind\ra\VE/\p}

% 124
\begin{inference-rule}{Nonrecursive, possibly simultaneous, bindings}
\begin{equation}\label{rule:dynamic-core:valbind1}
\vcenter{\infer{\E\ts\longvalbind\ra\VE\ \optional{ +\ \VE'}}
  {\E\ts\exp\ra\V
    & \E,\V\ts\pat\ra\VE
    & \optional{\E\ts\valbind\ra\VE'}}}
\end{equation}
\end{inference-rule}

% 125
\begin{inference-rule}{Binding exception during declaration}
\begin{equation}\label{rule:dynamic-core:binding-exception}
\vcenter{\infer{\E\ts\longvalbind\ra[\ml{Bind}]}
  {\E\ts\exp\ra\V
    & \E,\V\ts\pat\ra\FAIL}}
\end{equation}
\end{inference-rule}

% 126
\begin{inference-rule}{Recursive value binding}
\begin{equation}\label{rule:dynamic-core:recursive-value-binding}
\vcenter{\infer{\E\ts\recvalbind\ra\Rec\VE}
  {\E\ts\valbind\ra\VE}}
\end{equation}
\end{inference-rule}

\rulesec{Type Bindings}{\ts\typbind\ra\TE}
% 127
\begin{inference-rule}{Type binding}
\begin{equation}\label{rule:dynamic-core:type-binding}
\vcenter{\infer{\ts\longtypbind\ra\{\tycon\mapsto\emptymap\}\optional{+\TE}}
  {\optional{\ts\typbind\ra\TE}}}
\end{equation}
\end{inference-rule}

\rulesec{Datatype Bindings}{\ts\datbind\ra\VE,\TE}
% 128
\begin{inference-rule}{Datatype binding}
\begin{equation}\label{rule:dynamic-core:datatype-binding}
\vcenter{\infer{\ts\tyvarseq\;\tycon\boxml{=}\constrs\;\optional{\boxml{and}\,\datbind'}\ra\VE\optional{+\VE'},\{\tycon\mapsto\VE\}\optional{+\TE'}}
  {\ts\constrs\ra\VE
    & \optional{\ts\datbind'\ra\VE',\TE'}}}
\end{equation}
\end{inference-rule}

\rulesec{Constructor Bindings}{\ts\constrs\ra\VE}

% 129
\begin{inference-rule}{Constructor binding}
\begin{equation}\label{rule:dynamic-core:constructor-binding}
\vcenter{\infer{\ts\vid\optional{\boxml{|}\,\constrs}\ra
    \{\vid\mapsto(\vid,\isc)\}\,\optional{+\VE}}
  {\optional{\ts\constrs\ra\VE}}}
\end{equation}
\end{inference-rule}

\rulesec{Exception Bindings}{\E\ts\exnbind\ra\VE}

% 130
\begin{inference-rule}{Exception binding}
\begin{equation}\label{rule:dynamic-core:exception-binding1}
\vcenter{\infer{\s,\E\ts\longvidexnbindaa\ra\{\vid\mapsto(\e,\ise)\}\optional{ +\ \VE},\
    \s'\optional{'}}
  {\e\notin\of{\excs}{\s}
    & \s'=\s+\{\e\}
    & \optional{\s',\E\ts\exnbind\ra\VE,\s''} }}
\end{equation}
The two side conditions ensure that a new exception name is generated
and recorded as ``used'' in subsequent states.
\end{inference-rule}

% 131
\begin{inference-rule}{Looking up exception binding in environment}
When a $\longvid$ is bound to an exception.
\begin{equation}\label{rule:dynamic-core:exception-binding2}
\vcenter{\infer{\E\ts\longvidexnbindb\ra\{\vid\mapsto(\e,\ise)\}\optional{+\ \VE}}
  {\E(\longvid)=(\e,\ise)
    & \optional{\E\ts\exnbind\ra\VE} }}
\end{equation}
\end{inference-rule}

\rulesec{Atomic Patterns}{\E,\V\ts\atpat\ra\VE/\fail}

% 132
\begin{inference-rule}{Wildcard pattern}
\begin{equation}\label{rule:dynamic-core:wildcard-pattern}
\vcenter{\infer{\E,\V\ts\wildpat\ra \emptymap}
  {}}
\end{equation}
\end{inference-rule}

% 133
\begin{inference-rule}{Successfully match against special constant}
\begin{equation}\label{rule:dynamic-core:sconpat1}
\vcenter{\infer{\E,\V\ts\scon\ra \emptymap}
  {\V=\sconval(\scon)}}
\end{equation}
\end{inference-rule}

% 134
\begin{inference-rule}{Fail to match against special constant}
\begin{equation}\label{rule:dynamic-core:sconpat2}
\vcenter{\infer{\E,\V\ts\scon\ra \FAIL}
  {\V\neq\sconval(\scon)}}
\end{equation}
Any evaluation resulting in $\FAIL$ must do so because 
rule~\ref{rule:dynamic-core:sconpat2},
rule~\ref{rule:dynamic-core:value/exception-constructor-pattern-fail},
rule~\ref{rule:dynamic-core:conpat-fail},
or rule~\ref{rule:dynamic-core:exception-constructor-match-fail}
has been applied.
\end{inference-rule}

% 135
\begin{inference-rule}{Variable pattern}
\begin{equation}\label{rule:dynamic-core:variable-pattern}
\vcenter{\infer{\E,\V\ts\vid\ra \{\vid\mapsto(\V,\isv)\} }
  {\hbox{$\vid\notin\Dom(\E)$ or $\of{\is}{\E(\vid)} = \isv$}}}
\end{equation}
\end{inference-rule}

% 136
\begin{inference-rule}{Value/exception constructor pattern}
\begin{equation}\label{rule:dynamic-core:value/exception-constructor-pattern}
\vcenter{\infer{\E,\V\ts\longvid\ra\emptymap}
  {\E(\longvid)=(\V,\is)
    & \is\neq\isv}}
\end{equation}
\end{inference-rule}

% 137
\begin{inference-rule}{Value/exception constructor pattern fail}
\begin{equation}\label{rule:dynamic-core:value/exception-constructor-pattern-fail}
\vcenter{\infer{\E,\V\ts\longvid\ra\FAIL}
  {\E(\longvid)=(\V',\is)
    &\is\neq\isv
    &\V\neq\V'}}
\end{equation}
Any evaluation resulting in $\FAIL$ must do so because 
rule~\ref{rule:dynamic-core:sconpat2},
rule~\ref{rule:dynamic-core:value/exception-constructor-pattern-fail},
rule~\ref{rule:dynamic-core:conpat-fail},
or rule~\ref{rule:dynamic-core:exception-constructor-match-fail}
has been applied.
\end{inference-rule}

% 138
\begin{inference-rule}{Record pattern}
\begin{equation}\label{rule:dynamic-core:record-pattern}
\vcenter{\infer{\E,\V\ts\lttbrace\ \optional{\labpats}\ \rttbrace\ra\emptymap\optional{+\VE/\fail}}
    {\V=\emptymap\optional{+\r}\ \In\ \Val
      & \optional{\E,\r\ts\labpats\ra\VE/\fail}}}
\end{equation}
\end{inference-rule}

% 139
\begin{inference-rule}{Parenthesised pattern}
\begin{equation}\label{rule:dynamic-core:parenthesised-pattern}
\vcenter{\infer{\E,\V\ts\parpat\ra\VE/\fail}
  {\E,\V\ts\pat\ra\VE/\fail}}
\end{equation}
\end{inference-rule}

\rulesec{Pattern Rows}{\E,\r\ts\labpats\ra\VE/\fail}

% 140
\begin{inference-rule}{Wildcard record}
\begin{equation}\label{rule:dynamic-core:wildcard-record}
\vcenter{\infer{\E,\r\ts\wildrec\ra\emptymap}
  {}}
\end{equation}
\end{inference-rule}

% 141
\begin{inference-rule}{Record component with inherited FAIL}
\begin{equation}\label{rule:dynamic-core:longlab-dyn-rule1}
\vcenter{\infer{\E,\r\ts\longlabpats\ra\FAIL}
  {\E,\r(\lab)\ts\pat\ra\FAIL}}
\end{equation}
For well-typed programs $\lab$ will be in the domain of $\r$.
\end{inference-rule}

% 142
\begin{inference-rule}{Record component}
\begin{equation}\label{rule:dynamic-core:record-component}
\vcenter{\infer{\E,\r\ts\longlabpats\ra
    \VE\optional{ +\ \VE'/\fail}}
  {\E,\r(\lab)\ts\pat\ra\VE
    & \optional{\E,\r\ts\labpats\ra\VE'/\fail} }}
\end{equation}
For well-typed programs $\lab$ will be in the domain of $\r$.
\end{inference-rule}

\rulesec{Patterns}{\E,\V\ts\pat\ra\VE/\fail}

% 143
\begin{inference-rule}{Atomic pattern}
\begin{equation}\label{rule:dynamic-core:atomic-pattern}
\vcenter{\infer{\E,\V\ts\atpat\ra \VE/\fail}
  {\E,\V\ts\atpat\ra \VE/\fail}}
\end{equation}
\end{inference-rule}

% 144
\begin{inference-rule}{Construction pattern}
\begin{equation}\label{rule:dynamic-core:conpat-rule1}
\vcenter{\infer{\E,\V\ts\vidpat\ra \VE/\fail}
  {\begin{array}{c}
      \E(\longvid) = (\vid,\isc)\qquad\vid\neq\REF\qquad
      \V=(\vid,\V')\\
      \E,\V'\ts\atpat\ra\VE/\fail
\end{array}}}
\end{equation}
\end{inference-rule}

% 145
\begin{inference-rule}{Construction pattern failure}
\begin{equation}\label{rule:dynamic-core:conpat-fail}
\vcenter{\infer{\E,\V\ts\vidpat\ra \FAIL}
  {\E(\longvid) = (\vid,\isc)
    & \vid\neq\REF
    & \V\notin\{\vid\}\times\Val}}
\end{equation}
Any evaluation resulting in $\FAIL$ must do so because 
rule~\ref{rule:dynamic-core:sconpat2},
rule~\ref{rule:dynamic-core:value/exception-constructor-pattern-fail},
rule~\ref{rule:dynamic-core:conpat-fail},
or rule~\ref{rule:dynamic-core:exception-constructor-match-fail}
has been applied.
\end{inference-rule}

% 146
\begin{inference-rule}{Exception constructor match}
\begin{equation}\label{rule:dynamic-core:exception-constructor-match}
\vcenter{\infer{\E,\V\ts\vidpat\ra\VE/\FAIL}
  {\begin{array}{c}
      \E(\longvid)=(\e,\ise)\qquad\V=(\e,\V')\\
      \E,\V'\ts\atpat\ra\VE/\FAIL
      \end{array}
}}
\end{equation}
\end{inference-rule}

% 147
\begin{inference-rule}{Exception constructor match fail}
\begin{equation}\label{rule:dynamic-core:exception-constructor-match-fail}
\vcenter{\infer{\E,\V\ts\vidpat\ra\FAIL}
  {\E(\longvid)=(\e,\ise)
    & \V\notin\{\e\}\times\Val}}
\end{equation}
Any evaluation resulting in $\FAIL$ must do so because 
rule~\ref{rule:dynamic-core:sconpat2},
rule~\ref{rule:dynamic-core:value/exception-constructor-pattern-fail},
rule~\ref{rule:dynamic-core:conpat-fail},
or rule~\ref{rule:dynamic-core:exception-constructor-match-fail}
has been applied.
\end{inference-rule}

% 148
\begin{inference-rule}{Reference pattern}
\begin{equation}\label{rule:dynamic-core:reference-pattern}
\vcenter{\infer{\s,\E,\A\ts\REF\ \atpat\ra \VE/\fail,\s}
  {\s(\A)=\V
    & \s,\E,\V\ts\atpat\ra\VE/\fail,\s}}
\end{equation}
\end{inference-rule}

% 149
\begin{inference-rule}{Layered pattern}
\begin{equation}\label{rule:dynamic-core:layered-pattern}
\vcenter{\infer{\E,\V\ts\layeredvidpata\ra\{\vid\mapsto(\V,\isv)\}+\VE/\fail}
  {\E,\V\ts\pat\ra\VE/\fail}}
\end{equation}
\end{inference-rule}


