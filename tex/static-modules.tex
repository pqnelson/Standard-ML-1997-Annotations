\chapter{Static Semantics for Modules}

\section{Semantic Objects}

\begin{clause}{Simple semantic objects}
The simple semantic objects for Modules static semantics are exactly
those for the Core from Section~\ref{sec:static-core:simple-objects}.
\end{clause}

\begin{definition}{Compound semantic objects}
The compound objects are those for Core (Section~\ref{sec:static-core:compound-objects}),
augmented by the following:
\begin{align*}
\sig\ {\rm or}\ \newlongsig{}
                & \in    \Sig =  \TyNameSets\times\Env \\
\funsig\ {\rm or}\ \newlongfunsig{}
                & \in    \FunSig = \TyNameSets\times
                                         (\Env\times\Sig)\\
\G              & \in    \SigEnv        =       \finfun{\SigId}{\Sig} \\
\F              & \in    \FunEnv        =       \finfun{\FunId}{\FunSig} \\
\B\ {\rm or}\ \T,\F,\G,\E
                & \in    \Basis = \TyNameSets\times
                                              \FunEnv\times\SigEnv\times\Env
\end{align*}
The 
prefix $(\T)$, in signatures and functor signatures, binds  type names.
\end{definition}

\begin{clause}{Projection, injection, modification}
These operations are analogous `in the obvious way' to how they were
defined for Core's static semantics.
\end{clause}

\begin{definition}{Context of basis}
The Definition introduces the notation $\of{\C}{\B}$ to be the context
$(\of{\T}{\B},\emptyset,\of{\E}{\B})$, i.e.~with an empty set of
explicit type variables.
\end{definition}

\begin{definition}{Modifying a basis by an environment}
We frequently need to modify a basis $\B$ by an environment $\E$
(or a structure environment $\SE$ say),
at the same time extending $\of{\T}{\B}$ to include the type names.
We therefore define $\B\oplus\SE$, for example, to mean
$\B+(\TyNamesFcn\SE,\SE)$.
\end{definition}

\begin{clause}{Structures not represented by semantic objects}
There is no separate kind of semantic object to represent structures:
structure expressions elaborate to environments, just as structure-level
declarations do. Thus, notions which are commonly associated with
structures (for example the notion of matching a structure against a
signature) are defined in terms of environments.
\end{clause}

\section{Type Realisation}

\begin{definition}{Type realisation, or simply just a ``realisation''}
We define a \define{Type Realisation} is a map
$\rea:\TyNames\to\TypeFcn$
such that
$\t$ and $\rea(\t)$ have the same arity, and
if $t$ admits equality then so does $\rea(\t)$.
\end{definition}

\begin{definition}{Support of a type realisation}
The \define{Support} $\Supp\rea$ of a type realisation $\rea$ is the set of
type names $\t$ for which $\rea(\t)\ne\t$.
\end{definition}

\begin{definition}{Yield of a type realisation}
The \define{Yield} $\Yield\rea$ of a realisation $\rea$ is the set of
type names which occur in some $\rea(\t)$ for which $\t\in\Supp\rea$.
\end{definition}

\begin{clause}{Extend realisations to all semantic objects}
Realisations $\rea$ are extended to apply to all semantic objects; their
effect is to
replace each name $\t$ by $\rea(\t)$. In applying $\rea$ to an object with
bound names, such as a signature $\newlongsig{}$, first bound names must be
changed so that, for each binding prefix $(\T)$,
\begin{equation*}
\T\cap(\Supp\rea\cup\Yield\rea)=\emptyset.
\end{equation*}
\end{clause}

\section{Signature Instantiation}

\begin{definition}{}
An environment $\E_2$ is an \define{Instance} of a signature
$\sig_1=\newlongsig{1}$,
written $\siginst{\sig_1}{}{\E_2}$, if there exists a realisation
$\rea$
such that $\rea(\E_1)=\E_2$ and $\Supp\rea\subseteq\T_1$.
\end{definition}

\section{Functor Signature Instantiation}

\begin{definition}{}
A pair $(\E,(\T')\E')$ is called a \define{Functor Instance}.
Given $\funsig=\newlongfunsig{1}$,
a functor instance $(\E_2,(\T_2')\E_2')$ is an \define{Instance} of
$\funsig$,
written $\funsiginst{\funsig}{}{(\E_2,(\T_2')\E_2')}$,
if there exists a realisation $\rea$
such that
$\rea(\E_1,(\T_1')\E_1')=(\E_2,(\T_2')\E_2')$ and
$\Supp\rea\subseteq\T_1$.
\end{definition}

\section{Enrichment}

\begin{clause}{}
In matching an environment to a signature, the environment will be
allowed both to have more components, and to be more polymorphic, than
(an instance of) the signature. More precisely, we define enrichment of
environments and type structures recursively as follows:
\end{clause}

\begin{definition}{Environment enrichment}
An environment $\E_{1}=\newlongE{1}$ \define{Enriches} another environment
$\E_{2} = \newlongE{2}$, written $\E_1\succ\E_2$,
if
\begin{enumerate}
\item $\Dom\SE_1\supseteq\Dom\SE_2$, and $\SE_1(\strid)\succ\SE_2(\strid)$
                                               for all $\strid\in\Dom\SE_2$
\item $\Dom\TE_1\supseteq\Dom\TE_2$, and $\TE_1(\tycon)\succ\TE_2(\tycon)$
                                               for all $\tycon\in\Dom\TE_2$
\item $\Dom\VE_1\supseteq\Dom\VE_2$, and $\VE_1(\vid)\succ\VE_2(\vid)$
                                               for all $\vid\in\Dom\VE_2$,
where $(\sigma_1,\is_1)\succ(\sigma_2,\is_2)$ means $\sigma_1\succ\sigma_2$ and
$$\is_1 = \is_2\quad\hbox{or}\quad \is_2 = \isv$$
\end{enumerate}
\end{definition}

\begin{definition}{Type structure enrichment}
Finally, a type structure $(\theta_{1},\VE_{1})$
\define{Enriches} another type structure $(\theta_{2},\VE_{2})$,
written $(\theta_{1},\VE_{1})\succ(\theta_{2},\VE_{2})$,
if
\begin{enumerate}
\item $\theta_1=\theta_2$
\item Either $\VE_{1}=\VE_{2}$ or $\VE_{2}=\emptymap$.
\end{enumerate}
\end{definition}

\section{Signature Matching}

\begin{definition}{Environment matches a signature}
An environment $\E$ \define{Matches} a signature $\sig_{1}$ if there
exists an environemnt $\E^{-}$ such that $\sig_{1}\geq E^{-}\prec E$.
Thus matching is a combination of instantiation and enrichment. There is
at most one such $\E^{-}$, given $\sig_{1}$ and $\E$.
\end{definition}

\section{Inference Rules}

\begin{clause}{Judgement form for static semantics of Modules}
The rules of the Modules static semantics allow
sentences of the form
\begin{equation*}
A\ts\phrase\ra A'
\end{equation*}
to be inferred, where in this case $A$ is either a basis, a context or
an environment and $A'$ is a semantic object.  The convention for options
is as in the Core semantics. 
\end{clause}

\begin{clause}{Assumption about basis for top-level declarations}
Although not assumed in our definitions, it is intended that every basis
$\B=\T,\F,\G,\E$ in which a $\topdec$ is elaborated has the property
that $\TyNamesFcn\F\ \cup\TyNamesFcn\G\cup\TyNamesFcn\E\subseteq\T$.
\end{clause}

\begin{theorem}{Static semantics preserves the assumption}
Let S be an inferred sentence $\B\ts\topdec\ra\B'$ in which $\B=\T,\F,\G,\E$ satisfies
the condition $\TyNamesFcn\F\ \cup\TyNamesFcn\G\cup\TyNamesFcn\E\subseteq\T$. Then $\B'$ also satisfies the condition.

Moreover, if S$'$ is a sentence of the form
$\B''\ts\phrase\ra A$ occurring in a proof of S, where $\phrase$ is
any Modules phrase, then $\B''$ also satisfies the condition.

Finally, if $\T,\U,\E\ts\phrase\ra A$ occurs
in a proof of S, where $\phrase$ is a phrase of Modules or of the Core, then
$\TyNamesFcn\E\subseteq\T$.
\end{theorem}

\rulesec{Structure Expressions}{\B\ts\strexp\ra \E}

\begin{inference-rule}{Generative structure expression}
\begin{equation}\label{rule:static-modules:generative-strexp}
  \vcenter{\infer{\B\ts\encstrexp\ra \E}
    {\B\ts\strdec\ra\E}}
\end{equation}
\end{inference-rule}

% 51
\begin{inference-rule}{Long structure identifier}
\begin{equation}\label{rule:static-modules:longstrid}
  \vcenter{\infer{\B\ts\longstrid\ra\S}
    {\B(\longstrid)=\S}}
\end{equation}
\end{inference-rule}

% 52
\begin{inference-rule}{Transparent constraint rule}
\begin{equation}\label{rule:static-modules:transparent-constraint-rule}
  \vcenter{\infer{\B\ts\transpconstraint\ra\E'}
    {\B\ts\strexp\ra\E
      &\B\ts\sigexp\ra\Sigma
      &\Sigma\geq\E'\prec\E}}
\end{equation}
\end{inference-rule}

% 53
\begin{inference-rule}{Opaque constraint rule}
\begin{equation}\label{rule:static-modules:opaque-constraint-rule}
  \vcenter{\infer{\B\ts\opaqueconstraint\ra\E'}
    {\deduce{\T' \cap(\of{\T}{\B}) = \emptyset}
        {\deduce{(\T')\E'\geq\E''\prec\E}
          {\deduce{\B\ts\sigexp\ra(\T')\E'}
            {B\ts\strexp\ra\E}}}}}
\end{equation}
\end{inference-rule}

% 54
\begin{inference-rule}{Functor application}
\begin{equation}\label{rule:static-modules:functor-application}
  \vcenter{\infer{\B\ts\funappstr\ra\E'}
    {\deduce{(\TyNamesFcn \E\; \cup\; \of{\T}{\B})\cap\T'=\emptyset}
      {\deduce{\funsiginst{\B(\funid)}{}{(\E'',(\T')\E')}\ ,
                                                    \ \E\succ\E''}
        {\B\ts\strexp\ra\E}}}}
\end{equation}
The side condition $(\TyNamesFcn\E \cup \of{\T}{\B})\cap\T'=\emptyset$
can always be satisfied by renaming bound names in $(\T')E'$; it ensures that the
generated datatypes receive new names.

Let $\B(\funid)=(\T)(\E_{f},(T')\E'_{f})$. Let $\rea$ be a realisation such
that $\rea(\E_{f},(T')\E'_{f})=(\E'',(\T')\E')$. Sharing between argument
and result specified in the declaration of the functor $\funid$ is
represented by the occurrence of the same name in both $\E_{f}$ and
$\E'_{f}$, and this repeated occurrence is preserved by $\rea$, yielding
sharing between the argument structure $\E$ and the result structure
$\E'$ of this functor application.
\end{inference-rule}

% 55
\begin{inference-rule}{Let-expression in structures}
\begin{equation}\label{rule:static-modules:let-strexp}
  \vcenter{\infer{\B\ts\letstrexp\ra\E_{2}}
    {\B\ts\strdec\ra\E_{1}
      &\B\oplus\E_{1}\ts\strexp\ra\E_{2}}}
\end{equation}
The use of $\oplus$, here and elsewhere, ensures that type names
generated by the first sub-phrase are distinct from names generated by
the second sub-phrase.
\end{inference-rule}

\rulesec{Structure-level Declarations}{\B\ts\strdec\ra\E}

% 56
\begin{inference-rule}{Core declaration}
\begin{equation}\label{rule:static-modules:core-declaration}
  \vcenter{\infer{\B\ts\dec\ra\E}
    {\of{\C}{\B}\ts\dec\ra\E}}
\end{equation}
\end{inference-rule}

% 57
\begin{inference-rule}{Structure declaration}
\begin{equation}\label{rule:static-modules:structure-declaration}
  \vcenter{\infer{\B\ts\singstrdec\ra\SE\ \In\ \Env}
    {\B\ts\strbind\ra\SE}}
\end{equation}
\end{inference-rule}

% 58
\begin{inference-rule}{Local structure-level declaration}
\begin{equation}\label{rule:static-modules:local-structure-level-declaration}
  \vcenter{\infer{\B\ts\localstrdec\ra\E_{2}}
    {\B\ts\strdec_{1}\ra\E_{1}
      & \B\oplus\E_{1}\ts\strdec_{2}\ra\E_{2}}}
\end{equation}
\end{inference-rule}

% 59
\begin{inference-rule}{Empty declaration}
\begin{equation}\label{rule:static-modules:empty-strdec-rule}
  \vcenter{\infer{\B\ts\emptystrdec\ra \emptymap{\rm\ in}\ \Env}
    {}}
\end{equation}
\end{inference-rule}

\begin{inference-rule}{Sequential declaration}
\begin{equation}\label{rule:static-modules:sequential-strdec-rule}
  \vcenter{\infer{\B\ts\seqstrdec\ra\plusmap{\E_{1}}{\E_{2}}}
    {\B\ts\strdec_{1}\ra\E_{1}
      & \B\oplus\E_{1}\ts\strdec_{2}\ra\E_{2}}}
\end{equation}
\end{inference-rule}

\rulesec{Structure Bindings}{\B\ts\strbind\ra\SE}

\begin{inference-rule}{Structure binding}
\begin{equation}\label{rule:static-modules:structure-binding}
  \vcenter{\infer{\B\ts\barestrbindera\ra\{\strid\mapsto\E\}
       \ \optional{ +\ \SE}}
    {\B\ts\strexp\ra\E
      & \optional{\plusmap{\B}{\TyNamesFcn\E}\ts\strbind\ra\SE}}}
\end{equation}
\end{inference-rule}

\rulesec{Signature Expressions}{\B\ts\sigexp\ra\E}

\begin{inference-rule}{Encapsulation signature expression}
\begin{equation}\label{rule:static-modules:encapsulating-sigexp}
  \vcenter{\infer{\B\ts\encsigexp\ra  \E}
    {\B\ts\spec\ra\E}}
\end{equation}
\end{inference-rule}

\begin{inference-rule}{Signature identifier}
\begin{equation}\label{rule:static-modules:signature-identifier}
  \vcenter{\infer{\B\ts\sigid\ra\E}
    {\B(\sigid) = (\T)\E
      & \T\cap (\of{\T}{\B}) = \emptyset}}
\end{equation}
The bound names of $\B(\sigid)$ can always be renamed to satisfy $\T\cap(\of{\T}{\B}) = \emptyset$,
if necessary.
\end{inference-rule}


\begin{inference-rule}{``Where type''}
Note this rule has $3+2+3=8$ premises.
\begin{equation}\label{rule:static-modules:where-type}
\vcenter{\infer{\B\ts\wheretypesigexp\ra\rea(\E)}
  {\deduce{\rea = \{\t\mapsto \Lambda\alphak.\tau\}\qquad
      \hbox{$\Lambda\alphak.\tau$ admits equality, if $\t$ does
        \qquad $\rea(\E)$ well-formed}}
    {\deduce{\E(\longtycon) = (\t, \VE)
        \qquad t\notin\of{\T}{\B}}
      {\B\ts\sigexp\ra \E
        \qquad \tyvarseq = \alphak
        \qquad \of{\C}{\B}\ts\ty\ra \tau}}}}
\end{equation}
\end{inference-rule}

\rulesec{}{\B\ts\sigexp\ra\sig}

\begin{danger}
The Definition has apparently introduced a new subsection consisting of
a single rule, but has not named the subsection. This was particularly
confusing to me until I read the \LaTeX\ source for the Definition.
\end{danger}

\begin{inference-rule}{Topmost signature expression}
\begin{equation}\label{rule:static-modules:topmost-sigexp}
\vcenter{\infer{\B\ts\sigexp\ra (\T)\E}
  {\B\ts\sigexp\ra\E
    &\T= \TyNamesFcn\E\setminus(\of{\T}{\B})}}
\end{equation}
A signature expression $\sigexp$ which is an immediate constituent of
a signature binding, a signature constraint, or a
functor binding is elaborated to a signature, see rules
\ref{rule:static-modules:transparent-constraint-rule}, \ref{rule:static-modules:opaque-constraint-rule},
\ref{rule:static-modules:sigbind}, and \ref{rule:static-modules:functor-binding}.
\end{inference-rule}

\rulesec{Signature Declarations}{\B\ts\sigdec\ra\G}

% 66
\begin{inference-rule}{Single signature declaration}
\begin{equation}\label{rule:static-modules:single-sigdec}
\vcenter{\infer{\B\ts\singsigdec\ra\G}
  {\B\ts\sigbind\ra\G}}
\end{equation}
\end{inference-rule}

\rulesec{Signature Bindings}{\B\ts\sigbind\ra\G}

% 67
\begin{inference-rule}{Signature binding}
\begin{equation}\label{rule:static-modules:sigbind}
\vcenter{\infer{\B\ts\sigbinder\ra\{\sigid\mapsto\sig\}
       \ \optional{+\ \G}}
  {\B\ts\sigexp\ra\sig
    & \optional{\B\ts\sigbind\ra\G}}}
\end{equation}
\end{inference-rule}

\rulesec{Specifications}{\B\ts\spec\ra\E}

% 68
\begin{inference-rule}{Value specification}
\begin{equation}\label{rule:static-modules:value-specification}
\vcenter{\infer{\B\ts\valspec\ra\cl{}{\VE}\ \In\ \Env}
  {\of{\C}{\B}\ts\valdesc\ra\VE}}
\end{equation}
\VE{} is determined by \B{} and \valdesc.
\end{inference-rule}

% 69
\begin{inference-rule}{Type specification}
\begin{equation}\label{rule:static-modules:type-specification}
\vcenter{\infer{\B\ts\typespec\ra\TE\ \In\ \Env}
  {\of{\C}{\B}\ts\typdesc\ra\TE
    & \forall(\t,\VE)\in\Ran\TE,\hbox{\ $t$ does not admit equality}}}
\end{equation}
The typenames in \TE\ are new.
\end{inference-rule}

% 70
\begin{inference-rule}{Equality type specification}
\begin{equation}\label{rule:static-modules:eqtype-specification}
\vcenter{\infer{\B\ts\eqtypespec\ra\TE\ \In\ \Env}
  {\of{\C}{\B}\ts\typdesc\ra\TE
    & \forall(\t,\VE)\in \Ran\TE,\ \t {\rm\ admits\ equality}}}
\end{equation}
The typenames in \TE\ are new.
\end{inference-rule}

% 71
\begin{inference-rule}{Datatype specification}
\begin{equation}\label{rule:static-modules:data-specification}
\vcenter{\infer{\B\ts\datatypespec\ra(\VE,\TE)\ \In\ \Env}
  {\deduce{\hbox{$\TE$ maximises equality}}
    {\deduce{\of{\C}{\B}\oplus\TE\ts\datdesc\ra\VE,\TE}
      {\forall(\t,\VE')\in\Ran\TE, \t\notin\of{\T}{\B}}}}}
\end{equation}
The typenames in \TE\ are new.
\end{inference-rule}

% 72
\begin{inference-rule}{Datatype replication specification}
\begin{equation}\label{rule:static-modules:datatypereplspec}
\vcenter{\infer{\B\ts\datatypereplspec\ra (\VE,\TE)\ \In\ \Env}
    {\B(\longtycon) = (\typefcn,\VE)
      & \TE = \{\tycon\mapsto(\typefcn,\VE)\}}}
\end{equation}
\end{inference-rule}

% 73
\begin{inference-rule}{Exception specification}
\begin{equation}\label{rule:static-modules:exceptionspec}
\vcenter{\infer{\B\ts\exceptionspec\ra\VE\ \In\ \Env}
  {\of{\C}{\B}\ts\exndesc\ra\VE}}
\end{equation}
\VE\ is determined by \B\ and \exndesc\ and contains monotypes only.
\end{inference-rule}

% 74
\begin{inference-rule}{Structure specification}
\begin{equation}\label{rule:static-modules:structurespec}
\vcenter{\infer{\B\ts\structurespec\ra\SE\ \In\ \Env}
  {\B\ts\strdesc\ra\SE}}
\end{equation}
\end{inference-rule}

% 75
\begin{inference-rule}{Include signature specification}
\begin{equation}\label{rule:static-modules:include-signature}
\vcenter{\infer{ \B\ts\singleinclspec\ra\E }{  \B\ts\sigexp\ra\E}}
\end{equation}
\end{inference-rule}

% 76
\begin{inference-rule}{Empty specification}
\begin{equation}\label{rule:static-modules:emptyspec}
\vcenter{\infer{\B\ts\emptyspec\ra\emptymap{\rm\ in}\ \Env}
    {}}
\end{equation}
\end{inference-rule}

% 77
\begin{inference-rule}{Sequential specification}
\begin{equation}\label{rule:static-modules:seqspec}
\vcenter{\infer{\B\ts\seqspec\ra\plusmap{\E_{1}}{\E_{2}}}
  {\B\ts\spec_{1}\ra\E_{1}
    & \B\oplus\E_{1}\ts\spec_{2}\ra\E_{2}
    & \Dom(\E_{1})\cap\Dom(\E_{2}) = \emptyset}}
\end{equation}
Note that no sequential specification is allowed to specify the same
identifier twice.
\end{inference-rule}

% 78
\begin{inference-rule}{Sharing type specification}
Note this has $2+2+2=6$ premises.
\begin{equation}\label{rule:static-modules:sharspec}
\vcenter{\infer{\B\ts \newsharingspec\ra \rea(\E)}
  {\deduce{\{\t_{1},\ldots,\t_{n}\}\cap\of{\T}{\B} = \emptyset\qquad
           \rea = \{\t_{1}\mapsto\t,\ldots,\t_{n}\mapsto \t\}}
    {\deduce{t\in\{\t_{1},\ldots,\t_{n}\}\qquad\hbox{$\t$ admits equality, if some $\t_{i}$ does}}
      {\B\ts \spec \ra \E\qquad \E(\longtycon_{i}) = (\t_{i},\VE_{i}), \;i = 1..n}}}}
\end{equation}
\end{inference-rule}

\rulesec{Value Descriptions}{\C\ts\valdesc\ra\VE}
% 79
\begin{inference-rule}{Value description}
\begin{equation}\label{rule:static-modules:valdesc}
\vcenter{\infer{\C\ts\valviddescription\ra\{\vid\mapsto(\tau,\isv)\}
       \ \optional{ +\ \VE}}
  {\C\ts\ty\ra\tau
    & \optional{\C\ts\valdesc\ra\VE}}}
\end{equation}
\end{inference-rule}

\rulesec{Type Descriptions}{\C\ts\typdesc\ra\TE}
% 80
\begin{inference-rule}{Type description}
\begin{equation}\label{rule:static-modules:typedesc}
\vcenter{\infer{\C\ts\typdescription\ra\{\tycon\mapsto(\t,\emptymap)\}
       \ \optional{ +\ \TE}}
  {\deduce{\tyvarseq = \alphak\quad\t\notin\of{\T}{\C}\quad\arity\t=k}
    {\optional{\C\ts\typdesc\ra\TE\qquad\t\notin\TyNamesFcn\TE}}}}
\end{equation}
Note that the value environment in the resulting type structure must be
empty. For example, \mbox{\ml{datatype s=C type t}} \mbox{\ml{sharing type t=s}}\  
is a legal specification, but the type structure bound to \ml{t}
does not bind any value constructors.
\end{inference-rule}

\rulesec{Datatype Descriptions}{\C\ts\datdesc\ra\VE,\TE}
% 81
\begin{inference-rule}{Datatype description}
There are $2+3=5$ premises to this rule.
\begin{equation}\label{rule:static-modules:datdesc}
  \vcenter{\infer{\begin{array}{l}\C\ts\newdatdescription\ra\\
    \qquad\cl{}{\VE} \optional{ +\ \VE'},\
       \{\tycon\mapsto(t,\cl{}{\VE})\}\ \optional{ +\ \TE'}
      \end{array}}
  {\deduce{\tyvarseq = \alphak\qquad\C,\alphakt\ts\condesc\ra\VE
           \quad\arity \t  =  k}
    {\optional{\C\ts\datdesc'\ra\VE',\TE'\qquad \forall(\t',\VE'')\in\Ran\TE',\,\t\neq\t'}}}}
\end{equation}
\end{inference-rule}

\rulesec{Constructor Descriptions}{\C,\tau\ts\condesc\ra\VE}
% 82
\begin{inference-rule}{}
\begin{equation}\label{rule:static-modules:}
\vcenter{\infer{\begin{array}{l}
      \C,\tau\ts\longconviddescription\ra\\
      \qquad\{\vid\mapsto(\tau,\isc)\}\
     \optional{ +\ \{\vid\mapsto(\tau'\to\tau,\isc)\}\ }\
      \optional{\optional{ +\ \VE}}
      \end{array}}
  {\optional{\C\ts\ty\ra\tau'}
    & \optional{\optional{\C,\tau\ts\condesc\ra\VE}}}}
\end{equation}
\end{inference-rule}

\rulesec{Exception Descriptions}{\C\ts\exndesc\ra\VE}
% 83
\begin{inference-rule}{Exception description}
\begin{equation}\label{rule:static-modules:exndesc}
\vcenter{\infer{\begin{array}{l}
        \C\ts\exnviddescriptiona\ra\\
        \quad\quad\{\vid\mapsto(\EXCN,\ise)\}\ \optional{ +\ \{\vid\mapsto(\tau\rightarrow\EXCN,\ise)\}}\ \optional{\optional{ +\ \VE}} 
       \end{array}}
  {\optional{\C\ts\ty\ra\tau\qquad\TyVarsFcn(\tau)=\emptyset}
    & \optional{\optional{\C\ts\exndesc\ra\VE}}}}
\end{equation}
\end{inference-rule}

\rulesec{Structure Descriptions}{\B\ts\strdesc\ra\SE}
% 84
\begin{inference-rule}{Structure description}
\begin{equation}\label{rule:static-modules:strdesc}
\vcenter{\infer{\B\ts\strdescription\ra\{\strid\mapsto\E\}\ \optional{ +\ \SE}}
  {\B\ts\sigexp\ra\E
    & \optional{\B + \TyNamesFcn\E\ts\strdesc\ra\SE}}}
\end{equation}
\end{inference-rule}

\rulesec{Functor Declarations}{\B\ts\fundec\ra\F}

% 85
\begin{inference-rule}{Single functor declaration}
\begin{equation}\label{rule:static-modules:singfundec}
  \vcenter{\infer{ \B\ts\singfundec\ra\F }
    { \B\ts\funbind\ra\F }}
\end{equation}
\end{inference-rule}

\rulesec{Functor Bindings}{\B\ts\funbind\ra\F}
% 86
\begin{inference-rule}{Functor binding}
\begin{equation}\label{rule:static-modules:functor-binding}
  \vcenter{\infer{\begin{array}{c}
        \B\ts\barefunstrbinder\ \optional{\mbox{\texttt{and} \funbind}}\ra\\
       \qquad\qquad \qquad
              \{\funid\mapsto(\T)(\E,(\T')\E')\}
              \ \optional{\optional{ +\ \F}}
      \end{array}}
  {\deduce{\optional{\B\ts\funbind\ra\F}}
    {\deduce{\T\cap(\of{\T}{\B}) = \emptyset\quad \T' = \TyNamesFcn\E'\setminus((\of{\T}{\B})\cup\T)}
      {\B\ts\sigexp\ra(\T)\E\qquad
      \B\oplus\{\strid\mapsto\E\} \ts\strexp\ra\E'}}}}
\end{equation}
Since $\oplus$ is used, any type name $\t$ in $\E$ acts like a constant in the functor body; in particular,
it ensures that further names generated during elaboration of the
body are distinct from $\t$. The set $\T'$ is
chosen such that every  name free
in $(\T)\E$ or $(\T)(\E,(\T')\E')$ is free in $\B$.
\end{inference-rule}

\rulesec{Top-level Declarations}{\B\ts\topdec\ra\B'}

% 87
\begin{inference-rule}{Structure-level declaration}
\begin{equation}\label{rule:static-modules:strdectopdec}
\vcenter{\infer{\B\ts\strdecintopdec\ra\B''}
    {\begin{array}{c}
        \B\ts\strdec\ra\E \qquad \optional{\B\oplus \E\ts \topdec\ra \B'}\\
        B'' = (\TyNamesFcn\E,\E)\ \In\ \Basis\; \optional{ + \B'}\qquad\TyVarFcn\B''=\emptyset
      \end{array}}}
\end{equation}
No free type variables enter the  basis: if $\B\ts\topdec\ra\B'$
then $\TyVarsFcn(\B') = \emptyset$.
\end{inference-rule}

% 88
\begin{inference-rule}{Signature Declaration}
\begin{equation}\label{rule:static-modules:sigdectopdec}
\vcenter{\infer{\B\ts\sigdecintopdec\ra \B''}
  {\begin{array}{c}
      \B\ts\sigdec\ra \G\qquad \optional{\B\oplus\G\ts\topdec\ra\B'}\\
      \B'' = (\TyNamesFcn\G,G)\ \In\ \Basis\;\langle + \B'\rangle
\end{array}}}
\end{equation}
No free type variables enter the  basis: if $\B\ts\topdec\ra\B'$
then $\TyVarsFcn(\B') = \emptyset$.
\end{inference-rule}

% 89
\begin{inference-rule}{Functor declaration}
\begin{equation}\label{rule:static-modules:fundectopdec}
\vcenter{\infer{\B\ts\fundecintopdec\ra\B''}
  {\begin{array}{c}
      \B\ts\fundec\ra\F\qquad\optional{\B\oplus\F\ts\topdec\ra\B'}\\
      B'' = (\TyNamesFcn\F,\F)\ \In\ \Basis\; \optional{+\B'}\qquad \TyVarsFcn\B''=\emptyset
      \end{array}}}
\end{equation}
No free type variables enter the  basis: if $\B\ts\topdec\ra\B'$
then $\TyVarsFcn(\B') = \emptyset$.
\end{inference-rule}

%% \begin{inference-rule}{}
%% \begin{equation}\label{rule:static-modules:}
%% \vcenter{\infer{}
%%    {}}
%% \end{equation}
%% \end{inference-rule}