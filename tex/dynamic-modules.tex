\chapter{Dynamic Semantics for Modules}\label{ch:dynamic-modules}

\section{Reduced Syntax}

\begin{clause}{}
The syntax for Modules is reduced in the following manner, \emph{in addition}
to the reduced syntax made for Core.
\end{clause}

\begin{clause}{Omit qualifications from descriptions}
Qualifications ``\texttt{of} $\ty$'' are omitted from constructor and
exception descriptions.
\end{clause}

\begin{clause}{Omit sharing/where type qualifications}
Any qualification ``\texttt{sharing type} $\syndots$'' on a specification,
or ``\texttt{where type} $\syndots$'' on a signature expression, both
are omitted. 
\end{clause}

\section{Compound Objects}\label{sec:dynamic-modules:compound-objects}

\begin{definition}{Compound objects}\label{defn:dynamic-modules:compound-objects}
The compound objects for the Modules dynamic semantics, in addition to
those for the Core dynamic semantics, are listed below:
\begin{equation*}
\begin{array}{rcl}
(\strid:\I,\strexp,\B)
                & \in   & \FunctorClosure = (\StrId\times\Int)\times\StrExp\times\Basis\\
%                &       & \qquad  = (\StrId\times\Int)\times\StrExp\times\Basis\\
\I\ {\rm or}\ (\SI,\TI,\VI)
                & \in   & \Int = \StrInt\times\TyInt\times\ValInt\\
        \SI     & \in   & \StrInt = \finfun{\StrId}{\Int}\\
        \TI     & \in   & \TyInt =  \finfun{\TyCon}{\ValInt}\\
        \VI     & \in   & \ValInt = \finfun{\VId}{\IdStatus}\\ 
        \G      & \in   & \SigEnv = \finfun{\SigId}{\Int}\\
        \F      & \in   & \FunEnv = \finfun{\FunId}{\FunctorClosure}\\
(\F,\G,\E)\ {\rm or}\ \B
                & \in   & \Basis = \FunEnv\times\SigEnv\times\Env\\
(\G,\I)\ {\rm or}\ \IB
                & \in   & \IntBasis = \SigEnv\times\Int
\end{array}
\end{equation*}
\end{definition}

\begin{definition}{Interface}
An \define{Interface} $\I\in\Int$ represents a ``view'' of a structure.
\end{definition}


\begin{clause}{Signatures evaluate to interfaces}
Specifications and signature expressions will evaluate to interfaces;
moreover, during the evaluation of a specification or signature expression, 
structures (to which a specification or signature expression may
refer via datatype replicating specifications) are represented 
only by their interfaces.
\end{clause}

\begin{definition}{Extracting an interface from an environment}
To extract a value interface from
a dynamic value environment we define the operation $\Inter: \ValEnv \to\ValInt$
as follows:
\begin{equation*}
\Inter(\VE) = \{\vid\mapsto\is\;;\;\VE(\vid) = (\V,\is)\}
\end{equation*}
In other words, $\Inter(\VE)$ is the value interface obtained from $\VE$ by
removing all values from $\VE$.

We then extend $\Inter$ to a function $\Inter:\Env\to\Int$ as follows:
\begin{equation*}
\Inter(\SE,\TE,\VE)\ =\ (\SI,\TI,\VI)
\end{equation*}
where $\VI$ = $\Inter(\VE)$ and 
\begin{align*}
\SI &= \{\strid\mapsto\Inter\E\;;\;\SE(\strid) = \E\}\\
\TI &= \{\tycon\mapsto\Inter\VE'\;;\;\TE(\tycon) = \VE'\}
\end{align*}
\end{definition}

\begin{definition}{Interface basis}
An \define{Interface Basis} $\IB=(\G,\I)$ is a value-free part of a
basis, sufficient to evaluate signature expressions and specifications.
\end{definition}


\begin{clause}{Extending Inter to create interface bases}
The function $\Inter$ is extended to create an interface basis
from a basis $\B$ as follows:
\begin{equation*}
\Inter(\F,\G,\E)\ =\ (\G, \Inter\E)
\end{equation*}
\end{clause}

\begin{definition}{Cutting down an environment to an interface}
A further operation
\[ \downarrow\ :\ \Env\times\Int\to\Env\]
is required, to cut down an environment $\E$ to a given interface $\I$,
representing the effect of an explicit signature ascription. We first
define $\downarrow: \ValEnv\times\ValInt\to\ValEnv$ by
\[\VE\downarrow\VI = \{\vid\mapsto(\V,\is)\;;\;\VE(\vid) = (\V,\is')\ {\rm and}\ \VI(\vid) = \is\}\]
(Note that $\VE$ and $\VI$ need not have the same
domain and that the identifier status is taken from $\VI$.) 
We then define $\downarrow: \StrEnv \times \StrInt \to \StrEnv$,
$\downarrow: \TyEnv \times\TyInt\to\TyEnv$ and
$\downarrow: \Env\times\Int\to\Env$ simultaneously as
follows:\footnote{Note: there is a typo in the third equation 
$(\SE,\TE,\VE)\downarrow(\SI,\TI,\VI)$
which the Definition, as printed, reads erroneously as
$(\SE,\TE,\VE)\downarrow(\SI,\TE,\VI)$.}
\label{downarrowdef}
\begin{equation*}
\SE\downarrow\SI  =  \{\strid\mapsto\E\downarrow\I\ ;\
          	\SE(\strid)=\E\ {\rm and}\ \SI(\strid)=\I\}
\end{equation*}
\begin{equation*}
\TE\downarrow\TI =  \{\tycon\mapsto \VE'\downarrow\VI'\ ;\ 
               \TE(\tycon) = \VE'\ {\rm and}\ \TI(\tycon) = \VI'\}
\end{equation*}
\begin{equation*}
(\SE,\TE,\VE)\downarrow(\SI,\TI,\VI)  = 
               (\SE\downarrow\SI, \TE\downarrow\TI, \VE\downarrow\VI)
\end{equation*}
\end{definition}

\begin{clause}{Static elaboration naturally implements interfaces}
It is important to note that an interface can also be obtained from the
\emph{static} value $\Sigma$ of a signature expression; 
it is obtained by first replacing every type
structure $(\typefcn, \VE)$
in the range of every type environment $\TE$ by $\VE$
and then replacing each pair $(\sigma,\is)$ in the range of every value
environment $\VE$ by $\is$.

Thus in an implementation interfaces would naturally be obtained from the
static elaboration; we choose to give separate rules here for obtaining them
in the dynamic semantics since we wish to maintain our separation of the
static and dynamic semantics, for reasons of presentation.
\end{clause}

\section{Inference Rules}

\begin{definition}{Judgement form}
The\index{59.2} semantic rules allow sentences  of the form
\begin{equation*}
\s,A\ts\phrase\ra A',\s'
\end{equation*}
to be inferred, where $A$ is either a basis, a signature environment, or
empty; $A'$ is some semantic object; and $s$, $s'$ are states before and
after the evaluation represented by the judgement.
\end{definition}

\begin{definition}{Side-conditions}
Some hypotheses in rules are not of this form; they are called
\define{side-conditions}.
\end{definition}

\begin{convention}{Options same as for Core}
The conventions for options are the same as for Core (those in single
brackets are all taken together, or none are taken together; separately,
those in double brackets are all taken together, or none are taken).
\end{convention}

\begin{convention}{State and exception conventions}\label{convention:dynamic-modules:state-and-exception}
The state convention~\zref{convention:dynamic-convention:state} and
exception convention~\zref{convention:dynamic-convention:exception} are
adopted as in the Core dynamics. However, it may be shown that the only
Modules phrases whose evaluation may cause a side-effect or generate an
exception packet are of the form $\strexp$, $\strdec$, $\strbind$ or
$\topdec$.
\end{convention}

\rulesec{Structure Expressions}{\B\ts\strexp\ra \E/\p}

% 150
\begin{inference-rule}{Generative structure expression}
\begin{equation}\label{rule:dynamic-modules:generative-structure-expression}
\vcenter{\infer{\B\ts\encstrexp\ra\E}
  {\B\ts\strdec\ra\E}}
\end{equation}
\end{inference-rule}

% 151
\begin{inference-rule}{Structure long id}
\begin{equation}\label{rule:dynamic-modules:longstrid}
\vcenter{\infer{\B\ts\longstrid\ra\E}
  {\B(\longstrid)=\E}}
\end{equation}
\end{inference-rule}

% 152
\begin{inference-rule}{Transparent signature constraint}
\begin{equation}\label{rule:dynamic-modules:transparent-signature-constraint}
\vcenter{\infer{\B\ts\transpconstraint\ra\E\downarrow\I}
  {\B\ts\strexp\ra\E
    & \Inter\B\ts\sigexp\ra\I}}
\end{equation}
\end{inference-rule}

% 153
\begin{inference-rule}{Opaque signature constraint}
\begin{equation}\label{rule:dynamic-modules:opaque-signature-constraint}
\vcenter{\infer{\B\ts\opaqueconstraint\ra\E\downarrow\I}
  {\B\ts\strexp\ra\E
    & \Inter\B\ts\sigexp\ra\I}}
\end{equation}
\end{inference-rule}

% 154
\begin{inference-rule}{Functor application}
\begin{equation}\label{rule:dynamic-modules:functor-application}
\vcenter{\infer{\B\ts\funappstr\ra\E'}
  {\begin{array}{c}
      \B(\funid)=(\strid:\I,\strexp',\B')\\
      \B\ts\strexp\ra\E\qquad
      \B'+\{\strid\mapsto\E\downarrow\I\}\ts\strexp'\ra\E'\\
   \end{array}}}
\end{equation}
Before the evaluation of the functor body $\strexp'$, the
actual argument $\E$ is cut down by the formal parameter
interface $\I$, so that any opening of $\strid$ resulting
from the evaluation of $\strexp'$ will produce no more components
than anticipated during the static elaboration.
\end{inference-rule}

% 155
\begin{inference-rule}{Let-structure expression}
\begin{equation}\label{rule:dynamic-modules:let-structure-expression}
\vcenter{\infer{\B\ts\letstrexp\ra\E'}
  {\B\ts\strdec\ra\E
    &\B+\E\ts\strexp\ra\E'}}
\end{equation}
\end{inference-rule}

\rulesec{Structure-level Declarations}{\B\ts\strdec\ra\E/\p}

% 156
\begin{inference-rule}{Core declaration}
\begin{equation}\label{rule:dynamic-modules:core-declaration}
\vcenter{\infer{\B\ts\dec\ra\E'}
  {\of{\E}{\B}\ts\dec\ra\E'}}
\end{equation}
\end{inference-rule}

\begin{remark}{For the ``Core fragment'' of ML}
We can ignore the next two rules, and use these structure-level
declarations rules to describe the dynamics of the ``Core'' fragment of
Standard ML.
\end{remark}

% 157
\begin{inference-rule}{Structure declaration}
\begin{equation}\label{rule:dynamic-modules:structure-declaration}
\vcenter{\infer{\B\ts\singstrdec\ra\SE\ \In\ \Env}
  {\B\ts\strbind\ra\SE}}
\end{equation}
\end{inference-rule}

% 158
\begin{inference-rule}{Local structure declaration}
\begin{equation}\label{rule:dynamic-modules:local-structure-declaration}
\vcenter{\infer{\B\ts\localstrdec\ra\E_{2}}
  {\B\ts\strdec_{1}\ra\E_{1}
    & \B+\E_{1}\ts\strdec_{2}\ra\E_{2}}}
\end{equation}
\end{inference-rule}

% 159
\begin{inference-rule}{Empty declaration}
\begin{equation}\label{rule:dynamic-modules:empty-declaration}
\vcenter{\infer{\B\ts\emptystrdec\ra \emptymap{\rm\ in}\ \Env}
  {}}
\end{equation}
\end{inference-rule}

% 160
\begin{inference-rule}{Sequential declaration}
\begin{equation}\label{rule:dynamic-modules:sequential-declaration}
\vcenter{\infer{\B\ts\seqstrdec\ra\plusmap{\E_{1}}{\E_{2}}}
  {\B\ts\strdec_{1}\ra\E_{1}
    & \B+\E_{1}\ts\strdec_{2}\ra\E_{2}}}
\end{equation}
\end{inference-rule}

\rulesec{Structure Bindings}{\B\ts\strbind\ra\SE/\p}

% 161
\begin{inference-rule}{Structure binding}
\begin{equation}\label{rule:dynamic-modules:structure-binding}
\vcenter{\infer{\B\ts\barestrbindera\ra\{\strid\mapsto\E\}
                \ \optional{ +\ \SE}}
  {\B\ts\strexp\ra\E
    & \optional{\B\ts\strbind\ra\SE}}}
\end{equation}
\end{inference-rule}

\rulesec{Signature Expressions}{\IB\ts\sigexp\ra\I}
% 162
\begin{inference-rule}{Encapsulation signature expressions}
\begin{equation}\label{rule:dynamic-modules:encapsulating-sigexp}
\vcenter{\infer{\IB\ts\encsigexp\ra\I}
  {\IB\ts\spec\ra\I}}
\end{equation}
\end{inference-rule}

% 163
\begin{inference-rule}{Signature identifier}
\begin{equation}\label{rule:dynamic-modules:signature-identifier}
\vcenter{\infer{\IB\ts\sigid\ra\I}
  {\IB(\sigid)=\I}}
\end{equation}
\end{inference-rule}

\rulesec{Signature Declarations}{\IB\ts\sigdec\ra\G}


% 164
\begin{inference-rule}{Single signature declaration}
\begin{equation}\label{rule:dynamic-modules:single-signature-declaration}
\vcenter{\infer{\IB\ts\singsigdec\ra\G}
  {\IB\ts\sigbind\ra\G}}
\end{equation}
\end{inference-rule}

\rulesec{Signature Bindings}{\IB\ts\sigbind\ra\G}

% 165
\begin{inference-rule}{Signature binding}
\begin{equation}\label{rule:dynamic-modules:signature-binding}
\vcenter{\infer{\IB\ts\sigbinder\ra\{\sigid\mapsto\I\}
       \ \optional{ +\ \G}}
  {\IB\ts\sigexp\ra\I
    &\optional{\IB\ts\sigbind\ra\G}}}
\end{equation}
\end{inference-rule}

\rulesec{Specifications}{\IB\ts\spec\ra\I}

% 166
\begin{inference-rule}{Value specification}
\begin{equation}\label{rule:dynamic-modules:value-specification}
\vcenter{\infer{\ts\valdesc\ra\VI}
  {\IB\ts\valspec\ra\VI\ \In\ \Int}}
\end{equation}
\end{inference-rule}

% 167
\begin{inference-rule}{Type specification}
\begin{equation}\label{rule:dynamic-modules:type-spec}
\vcenter{\infer{\IB\ts\typespec\ra\TI\ \In\ \Int}
  {\ts\typdesc\ra\TI}}
\end{equation}
\end{inference-rule}

% 168
\begin{inference-rule}{Equality type specification}
\begin{equation}\label{rule:dynamic-modules:eqtype-specification}
\vcenter{\infer{\IB\ts\eqtypespec\ra\TI\ \In\ \Int}
  {\ts\typdesc\ra\TI}}
\end{equation}
\end{inference-rule}

% 169
\begin{inference-rule}{Datatype specification}
\begin{equation}\label{rule:dynamic-modules:datatype-specification}
\vcenter{\infer{\IB\ts\datatypespec\ra(\VI,\TI)\ \In\ \Int}
  {\ts\datdesc\ra\VI,\TI}}
\end{equation}
\end{inference-rule}

% 170
\begin{inference-rule}{Datatype replication specification}
\begin{equation}\label{rule:dynamic-modules:datatype-replication-specification}
\vcenter{\infer{\IB\ts\datatypereplspec\ra(\VI,\TI)\ \In\ \Int}
  {\IB(\longtycon) = \VI
    & \TI = \{\tycon\mapsto\VI\}}}
\end{equation}
\end{inference-rule}

% 171
\begin{inference-rule}{Exception specification}
\begin{equation}\label{rule:dynamic-modules:exception-specification}
\vcenter{\infer{\IB\ts\exceptionspec\ra \VI\ \In\ \Int}
  {\ts\exndesc\ra\VI}}
\end{equation}
\end{inference-rule}

% 172
\begin{inference-rule}{Structure specification}
\begin{equation}\label{rule:dynamic-modules:structure-specification}
\vcenter{\infer{\IB\ts\structurespec\ra\SI\ \In\ \Int}
  {\IB\ts\strdesc\ra\SI}}
\end{equation}
\end{inference-rule}

% 173
\begin{inference-rule}{Include signature specification}
\begin{equation}\label{rule:dynamic-modules:include-signature-spec}
\vcenter{\infer{\IB\ts\singleinclspec\ra\I}
  {\IB\ts\sigexp\ra\I}}
\end{equation}
\end{inference-rule}

% 174
\begin{inference-rule}{Empty specification}
\begin{equation}\label{rule:dynamic-modules:empty-specification}
\vcenter{\infer{\IB\ts\emptyspec\ra\emptymap{\rm\ in}\ \Int}
  {}}
\end{equation}
\end{inference-rule}

% 175
\begin{inference-rule}{Sequential specification}
\begin{equation}\label{rule:dynamic-modules:sequential-spec}
\vcenter{\infer{\IB\ts\seqspec\ra\plusmap{\I_{1}}{\I_{2}}}
  {\IB\ts\spec_{1}\ra\I_{1}
    & \IB+\I_{1}\ts\spec_{2}\ra\I_{2}}}
\end{equation}
\end{inference-rule}

\rulesec{Value Descriptions}{\ts\valdesc\ra\VI}

% 176
\begin{inference-rule}{Value description}
\begin{equation}\label{rule:dynamic-modules:value-description}
\vcenter{\infer{\ts\vid\ \optional{\AND\ \valdesc}\ra
       \{\vid\mapsto\isv\}\ \optional{+\,\VI}}
  {\optional{\ts\valdesc\ra\VI}}}
\end{equation}
\end{inference-rule}

\rulesec{Type Descriptions}{\ts\typdesc\ra\TI}
% 177
\begin{inference-rule}{Type description}
\begin{equation}\label{rule:dynamic-modules:type-description}
\vcenter{\infer{\ts\typdescription\ra\{\tycon\mapsto\emptymap\}\optional{+\TI}}
  {\optional{\ts\typdesc\ra\TI}}}
\end{equation}
\end{inference-rule}

\rulesec{Datatype Descriptions}{\ts\datdesc\ra\VI, \TI}
% 178
\begin{inference-rule}{Datatype description}
\begin{equation}\label{rule:dynamic-modules:datatype-description}
\vcenter{\infer{\ts\datdescriptiona\ra\VI\,\optional{+\,\VI'}, \{\tycon\mapsto\VI\}\optional{+\TI'}}
  {\ts\condesc\ra\VI
    & \optional{\ts\datdesc'\ra\VI',\TI'}}}
\end{equation}
\end{inference-rule}

\rulesec{Constructor Descriptions}{\ts\condesc\ra\VI}
% 179
\begin{inference-rule}{Constructor description}
\begin{equation}\label{rule:dynamic-modules:constructor-description}
\vcenter{\infer{\ts\shortconviddesc\ra\{\vid\mapsto\isc\}\,\optional{+\VI}}
  {\optional{\ts\condesc\ra\VI}}}
\end{equation}
\end{inference-rule}

\rulesec{Exception Descriptions}{\ts\exndesc\ra\VI}
% 180
\begin{inference-rule}{}
\begin{equation}\label{rule:dynamic-modules:exception-description}
\vcenter{\infer{ \ts\vid\ \optional{\boxml{and\ }\exndesc}\ra\{\vid\mapsto\ise\}\ \optional{+ \VI} }
  { \optional{\ts\exndesc\ra\VI} }}
\end{equation}
\end{inference-rule}

\rulesec{Structure Descriptions}{\IB\ts\strdesc\ra\SI}
% 181
\begin{inference-rule}{Structure description}
\begin{equation}\label{rule:dynamic-modules:structure-description}
\vcenter{\infer{ \IB\ts\strdescription\ra\{\strid\mapsto\I\}\ \optional{ +\ \SI} }
  { \IB\ts\sigexp\ra\I
    & \optional{\IB\ts\strdesc\ra\SI} }}
\end{equation}
\end{inference-rule}

\rulesec{Functor Bindings}{\B\ts\funbind\ra\F}
% 182
Note: we have fixed the typo reported by Rossberg~\cite{rossberg2018defects}.
\begin{inference-rule}{Functor binding}
\begin{equation}\label{rule:dynamic-modules:functor-binding}
\vcenter{\infer{\begin{array}{c}
      \B\ts\barefunstrbinder\ \optfunbinda\ra\\
      \qquad\qquad \qquad
      \{\funid\mapsto(\strid:\I,\strexp,\B)\}
      \ \optional{ +\ \F}
  \end{array}}
  {\Inter\B\ts\sigexp\ra\I
   & \optional{\B\ts\funbind\ra\F}}}
\end{equation}
\end{inference-rule}

\rulesec{Functor Declarations}{\B\ts\fundec\ra\F}
% 183
\begin{inference-rule}{Functor declaration}
\begin{equation}\label{rule:dynamic-modules:functor-declaration}
\vcenter{\infer{ \B\ts\singfundec\ra\F }
  { \B\ts\funbind\ra\F }}
\end{equation}
\end{inference-rule}

\rulesec{Top-level Declarations}{\B\ts\topdec\ra\B'/\p}

% 184
\begin{inference-rule}{Structure-level declaration}
Rossberg~\cite{rossberg2018defects} correctly identifies a typo in the
conclusion: the Definition writes ``$\B'\optional{'}$'', but what it
means is really ``$\B'\optional{+\B''}$''. We have fixed it here.
\begin{equation}\label{rule:dynamic-modules:structure-level-declaration}
\vcenter{\infer{\B\ts\strdecintopdec\ra\B'\optional{+\B''}}
  {\B\ts\strdec\ra\E
    &\B' =\E\ \In\ \Basis
    &\optional{ \B+\B'\ts\topdec\ra\B''} }}
\end{equation}
\end{inference-rule}

% 185
\begin{inference-rule}{Signature declaration}
Rossberg~\cite{rossberg2018defects} correctly identifies a typo in the
conclusion: the Definition writes ``$\B'\optional{'}$'', but what it
means is really ``$\B'\optional{+\B''}$''. We have fixed it here.
\begin{equation}\label{rule:dynamic-modules:signature-declaration}
\vcenter{\infer{\B\ts\sigdecintopdec\ra \B'\optional{+\B''}}
  {\Inter\B\ts\sigdec\ra\G
    & B' = \G\ \In\ \Basis
    & \optional{ \B + \B'\ts\topdec\ra\B''}}}
\end{equation}
\end{inference-rule}

% 186
\begin{inference-rule}{Functor [top] declaration}
Rossberg~\cite{rossberg2018defects} correctly identifies a typo in the
conclusion: the Definition writes ``$\B'\optional{'}$'', but what it
means is really ``$\B'\optional{+\B''}$''. We have fixed it here.
\begin{equation}\label{rule:dynamic-modules:functor-top-declaration}
\vcenter{\infer{\B\ts\fundecintopdec\ra\B'\optional{+\B''}}
  {\B\ts\fundec\ra\F
    & \B' = \F\ \In\ \Basis
    & \optional{ \B + \B'\ts\topdec\ra\B''}}}
\end{equation}
\end{inference-rule}