\chapter{Syntax of the Core}

\section{Reserved words}

\begin{definition}{Reserved words}\label{defn:reserved-words}
We are told the following are reserved words in the Core and may not be
used as identifiers (except for ``\texttt{=}''):
\vspace*{-6pt}
\begin{verbatim}
     abstype   and   andalso   as   case   datatype  do    else
     end    exception    fn    fun    handle    if   in   infix
     infixr   let     local    nonfix   of   op   open   orelse
     raise   rec   then   type   val   with   withtype    while
     (  )   [  ]   {  }  ,  :  ;  ...    _   |  =   =>   ->   #   
\end{verbatim}
\end{definition}

\begin{remark}{Ambiguity about reserved words}
Kahrs~\cite{kahrs1993mistakes} points out that there is ambiguity in the
notion of ``reserved words'', since there are reserved words \emph{in
the Core} and reserved words \emph{in Modules}. The Definition is not
clear with what it means by the term (at least, it is not clear to me).

I interpret ``reserved words'' to refer to those appearing \emph{either}
in Core (listed above) \emph{or} in Modules~\zref{defn:syntax-modules:reserved-words}.
\end{remark}

\section{Special Constants}\label{sec:ch2:special-constant}

\begin{definition}{Integer constant}
An \define{Integer constant} in decimal notation is ``an optional
negation symbol (\tttilde) followed by a non-empty sequence of decimal
digits \texttt{0}, \dots, \texttt{9}.'' Similarly, in hexadecimal
notation, an integer constant is an optional negation symbol followed by
\texttt{0x} followed by a non-empty sequence of hexadecimal digits
\texttt{0}, \dots, \texttt{9} and \texttt{a}, \dots, \texttt{f} (with
\texttt{A}, \dots, \texttt{F} also allowed).
\end{definition}

\begin{definition}{Word constant}
A \define{Word constant} in decimal notation is \texttt{0w} followed by
a nonempty sequence of decimal digits, and in hexadecimal notation
\texttt{0wx} followed by a non-empty sequence of hexadecimal digits.
\end{definition}

\begin{definition}{Real constant}
A \define{Real Constant} is an integer constant in decimal notation,
possibly followed by a point (\texttt{.}) and one or more decimal
digits, possibly followed by an exponent symbol (\texttt{E} or
\texttt{e}) and an integer constant in decimal notation; at least one of
these optional parts must occur, therefore no integer constant is a real
constant. 
\end{definition}

\begin{clause}{Assumption on the character set}
The Standard assumes an underlying alphabet of $N\geq256$ characters
numbered $0$ to $N-1$, which agrees with the ASCII character set on the
characters numbered $0$ to $127$.
\end{clause}

\begin{definition}{Ordinal range}
The \define{Ordinal range} of the alphabet is the closed interval $[0,N-1]$.
\end{definition}

\begin{definition}{String constant}
A \define{String constant} is a sequence, between quotation marks (\texttt{"})
of zero or more printable characters (i.e., numbered 33--126), spaces,
or escape sequences.

Each escape sequence starts with the character ``\verb+\+'' and stands
for a character sequence. The escape sequences are:
\halign{\indent#\hfil&\quad\parbox[t]{23pc}{\strut#\strut}\cr
\verb+\a+   & A single character interpreted by the system as alert (ASCII~7)\cr
\verb+\b+   & Backspace (ASCII 8)\cr
\verb+\t+   & Horizontal tab (ASCII 9)\cr
\verb+\n+   & Linefeed, also known as newline (ASCII 10)\cr
\verb+\v+   & Vertical tab (ASCII 11)\cr
\verb+\f+   & Form feed (ASCII 12)\cr
\verb+\r+   & Carriage return (ASCII 13)\cr
\verb+\^+$c$  & The control character $c$, where $c$ may
                be any character with number 64--95. The number
                of ~{\tt\char'134\char'136}$c$~ is 64 less than the 
                number of $c$.\cr
\verb+\+$ddd$ & The single character with number $ddd$ (3 decimal digits
denoting an integer in the ordinal range of the alphabet).\cr
\uconst & The single character with number $xxxx$ (4 hexadecimal digits
denoting an integer in the ordinal range of the alphabet).\cr
\verb+\"+   & {\tt "}\cr
\verb+\\+   & {\tt\char'134}\cr
\verb+\+$f\cdot\cdot f$\verb+\+
            & This sequence is ignored,
              where $f\cdot\cdot f$ stands for a sequence 
             of one or more formatting characters.\cr
}
\end{definition}

\begin{definition}{Formatting characters}\label{defn:syntax-core:formatting-characters}
The \define{Formatting Characters} are a subset of the non-printable
characters including at least space, tab, newline, and formfeed. The
formfeed allows long strings to be written on more than one line, by
writing ``~\verb+\+~'' at the end of one line and at the start of the next.
\end{definition}

\begin{definition}{Character constants}
A \define{Character constant} is a sequence of the form
\texttt{\#}$s$, where $s$ is a string constant denoting a string of
size one character.
\end{definition}

\begin{definition}{Special Constants}
The Definition denotes by {\SCon} the class of \define{Special Constants},
i.e., the integer, real, word, character, and string constants. The
definition uses {\scon} to range over \SCon.
\end{definition}

\section{Comments}

\begin{definition}{Comment}\label{defn:syntax-core:comment}
A \define{Comment} is any character sequence within comment brackets
``\texttt{(*}'' and ``\texttt{*)}'' in which comment brackets are
properly nested.
\end{definition}

\begin{clause}{Space illegal in comment brackets}
No space is allowed between the two characters which make up a comment
bracket --- i.e., no space is allowed in ``\texttt{(*}'' nor in
``\texttt{*)}'' if we want them to be comment brackets.
\end{clause}

\begin{clause}{Unterminated comments}
An unmatched ``\texttt{(*}'' should be detected by the compiler.
\end{clause}

\section{Identifiers}

\begin{definition}{Long identifiers, metavariable ``longx'' ranges over ``longX''}\label{defn:ch2:long-identifier}
For each syntactic class X marked as ``long'', there is a class
``longX'' of \define{Long Identifiers}. If \nonterminal{x} ranges over X, then
\nonterminal{longx} ranges over longX. The syntax of these long identifiers is given
by:
\begin{longtable}{rcl}
\nonterminal{longx} & $::=$ & \nonterminal{x} \\
& \alt & \nonterminal{strid${}_{1}$}\texttt{.}$\syndots$\texttt{.}\nonterminal{strid${}_{n}$}\texttt{.}\nonterminal{x}\\
& & ($n\geq1$)
\end{longtable}
\noindent for an identifier and a long identifier, respectively.

The qualified identifiers link the Core and Modules.
\end{definition}

\begin{clause}{Classes of identifiers}
There are five classes of identifiers:
\begin{enumerate}
\item Value Identifiers (long) --- \VId
\item Type Variables --- \TyVar
\item Type Constructors (long) --- \TyCon
\item Lab (record labels) --- \Lab
\item Structure Identifiers (long) --- \StrId
\end{enumerate}
\end{clause}

\begin{definition}{Valid identifiers}
An identifier is either:
\begin{enumerate}
\item an \define{Alphanumeric identifier} consisting of
any sequence of letters, digits, primes (\verb#'#), and underbars
(\verb#_#) starting with a letter or a prime; or
\item a \define{Symbolic identifier} consisting of any nonempty sequence
  of the following symbols: 
\vspace*{-6pt}
\begin{center}
%\verb(!  %  &  $  +  -  /  :  <  =  >  ?  (@\verb(  \  ~  `  ^  |  *(
\verb(!  %  &  $  #  +  -  /  :  <  =  >  ?  @  \  ~  `  ^  |  *(
\end{center}
\end{enumerate}
In both cases, reserved words are excluded.
\end{definition}

\begin{deviation}{Equality is not a valid identifier}
We will consciously deviate from the Definition and, consistent with our
choices made earlier, insist that ``\texttt{=}'' is not a valid symbolic
identifier. This deviates from the Definition, but it simplifies our
life formalizing the Definition.
\end{deviation}

\begin{definition}{Type Variables}
A \define{Type Variable} may be any alphanumeric identifier starting
with a prime. The class of type variables is denoted ``{\TyVar}''.
\end{definition}

\begin{definition}{Subclass of equality type variables}
The subclass of \define{Equality Type Variables} ETyVar of {\TyVar}
consists of those which start with two or more primes.
\end{definition}

\begin{convention}{VId, TyCon, Lab do not start with primes}
The convention therefore is that the classes \VId, \TyCon, and \Lab\ do
not start with primes.

Also, ``\texttt{*}'' is excluded from {\TyCon} to avoid confusion with
derived form of tuple types.
\end{convention}

\begin{definition}{Record labels and numeric labels}\label{defn:record-and-numeric-labels}
The syntactic class of \define{Record Labels}, denoted {\Lab}, is
the alphanumeric identifiers not starting with a prime, extended to
include the \define{Numeric Labels} \texttt{1}, \texttt{2}, \texttt{3},
and so on (i.e., any numeral not starting with \texttt{0}).
(This is so tuples can be syntactic sugar as a derived type.)
\end{definition}

\begin{definition}{Structure identifiers}
The identifier class of \define{Structure Identifiers}, denoted \StrId,
is represented by alphanumeric identifiers not starting with a prime.
\end{definition}

\begin{clause}{Determining class of identifier}
The class of type variables is disjoint from the other four classes. But
determining the class of an identifier \textit{id} in a Core phrase is
determined thus:
\begin{enumerate}
\item Immediately before ``\texttt{.}'' --- i.e., in a long identifier
  --- or in an \texttt{open} declaration, \textit{id} is a structure
  identifier. The rest of the cases assumes \textit{id} is not a structure identifier.
\item At the start of a component in a record type, record pattern, or
  record expression, \textit{id} is a record label.
\item Elsewhere in types, \textit{id} is a type constructor.
\item Elsewhere, \textit{id} is a value identifier.
\end{enumerate}
\end{clause}

\section{Lexical Analysis}

\begin{clause}{Items of analysis}
Each item of lexical analysis is either a reserved word
\zref{defn:reserved-words},
a numeric label~\zref{defn:record-and-numeric-labels}, 
a special constant (Sec~\ref{sec:ch2:special-constant}),
or a long identifier~\zref{defn:ch2:long-identifier}.
(Remember, an identifier \emph{is} a subclass of long identifiers.)
\end{clause}

\begin{clause}{Comments and formatting characters}
Comments~\zref{defn:syntax-core:comment} and formatting
characters~\zref{defn:syntax-core:formatting-characters} are separate
items (except, of course, within string constants) and are otherwise ignored.
\end{clause}

\begin{clause}{Longest next item is always taken}
At each stage, the longest next item is always taken.
\end{clause}

\section{Infixed operators}

\begin{definition}{Infix status and infixed operators}
An identifier may be given \define{Infix Status} by the \texttt{infix}
or \texttt{infixr} directive, which may occur as a declaration.
This status only pertains to its use as a \nonterminal{\vid} within the scope of the
directive, and in these uses it is called an \define{Infixed Operator}.
\end{definition}

\begin{clause}{Qualified identifiers never infixed}
Qualified identifiers \textbf{never} have infix status. The
Definition is very explicit about this.
\end{clause}

\begin{example}{Infixed operator}
If \nonterminal{\vid} has infix status, then ``$exp_{1}$
\nonterminal{\vid} $exp_{2}$'' (resp., 
``$pat_{1}$ \nonterminal{\vid} $pat_{2}$'') may occur --- possibly in parentheses ---
wherever the application ``\nonterminal{\vid}\verb+{+{\tt 1=}$\exp_{1}$\verb+,+{\tt
2=}$\exp_{2}$\verb+}+'' or its derived form ``\nonterminal{\vid}\verb+(+$exp_{1}$\verb+,+$exp_{2}$\verb+)+'' (resp.,
``\nonterminal{\vid}\verb+(+$pat_{1}$\verb+,+$pat_{2}$\verb+)+'') would otherwise
occur.
\end{example}

\begin{clause}{Op-prefix treats identifier as non-infixed}
An occurrence of \emph{any} long identifier (qualified or not) prefixed
by ``\texttt{op}'' is treated as non-infixed.

The only required use of an ``\texttt{op}'' is in prefixing a
non-infixed occurrence of an identifier \nonterminal{\vid} which has infix
status. Elsewhere ``\texttt{op}'', where permitted, has no effect.
\end{clause}

\begin{definition}{Nonfix directive}
Infix status is cancelled by the \texttt{nonfix} directive.
\end{definition}

\begin{definition}{Fixity directives}
The keywords ``\texttt{infix}'', ``\texttt{infixr}'', and ``\texttt{nonfix}''
are collectively referred to as \define{Fixity Directives}.
\end{definition}

\begin{clause}{Syntax of fixity directives}
For $n\geq1$, we have the syntax for fixity directives look like:
\[\mathtt{infix}\ \langle d\rangle\ vid_{1}\ \syndots\ vid_{n}\]
\[\mathtt{infixr}\ \langle d\rangle\ vid_{1}\ \syndots\ vid_{n}\]
\[\mathtt{nonfix}\ vid_{1}\ \syndots\ vid_{n}\]
where $\langle d\rangle$ is an optional decimal digit $d$ indicating
binding precedence. A higher value of $d$ indicates tighter binding; the
default is \texttt{0}.
\end{clause}

\begin{comment}{Precedence must be between 0 and 9}
Observe this part of the Definition explicit constrains the precedence
for infixed operators must be between 0 and 9. In principle, there's no
reason it cannot be something larger (say, representable as an unsigned
byte or word --- i.e., between 0 and 256, or between 0 and
$2^{64}\approx 18.4\times 10^{18}$).

For what it's worth, in \S4.4.2 of the Haskell 2010 report\footnote{\url{http://www.haskell.org/onlinereport/haskell2010/haskellch4.html\#x10-820004.4.2}}, it appears
that Haskell has the precedence [which it calls the ``fixity''] ``must be in the range 0 to 9''.
So Haskell is following Standard ML here, neat.
\end{comment}

\begin{clause}{Associativity of infixed operators}
Here ``\texttt{infix}'' dictates left associativity, ``\texttt{infixr}''
dictates right associativity.

The Definition requires in an expression of the form
$exp_{1}~vid_{1}~exp_{2}~vid_{2}~exp_{3}$ where $vid_{1}$ and $vid_{2}$
are infixed operators with the same precedence, either both must
associate to the left or both must associate to the right.
\end{clause}

\begin{example}{Infixed operator associativity}
Suppose ``\verb#<<#'' and ``\verb#>>#'' are infixed operators of equal
precedence, but ``\verb#<<#'' is left-associative and ``\verb#>>#'' is
right associative. Then
\tabskip4cm
\halign to\hsize{\indent\hfil{\tt #}\tabskip1em&\hfil#\hfil\ &\ {\tt #}\hfil\cr
x << y << z&parses as&(x << y) << z\cr
x >> y >> z&parses as&x >> (y >> z)\cr
x << y >> z&is illegal\cr
x >> y << z&is illegal\cr}
\end{example}

\begin{definition}{Scope of fixity directive}\label{syntax-core:infixed-operator:scope}
The \define{Scope} of a fixity directive $dir$ is the ensuing program
text, except if $dir$ occurs in a declaration $dec$ in either of the
phrases
\begin{equation*}
\mathtt{let}\ dec\ \mathtt{in}\ \syndots\ \mathtt{end}
\end{equation*}
or
\begin{equation*}
\mathtt{local}\ dec\ \mathtt{in}\ \syndots\ \mathtt{end}
\end{equation*}
then the scope of $dir$ does not extend beyond these phrases. Further
scope limitations are imposed for Modules.
\end{definition}

\begin{clause}{Do not appear in semantic rules}
The Definition states: These directives and ``\texttt{op}'' are omitted
from the semantic rules, since they only affect parsing.
\end{clause}

\section{Derived forms}

\begin{clause}{Syntactic sugar atop the bare language}
Many syntactic forms in Standard ML can be expressed in terms of a
smaller number of syntactic forms which the Definition calls the
\define{Bare} language. The derived forms and their equivalents are in
Appendix A.
\end{clause}

\section{Grammar}

\begin{convention}{Pseudo-EBNF syntax}
The Definition uses an idiosyncratic EBNF-like syntax. The rules
governing it:
\begin{enumerate}
\item The brackets \optional{\ } enclose optional phrases.
\item For any syntax class X (over which \nonterminal{x} ranges) we can define the
  class Xseq (over which \nonterminal{xseq} ranges) as follows:
\begin{longtable}{rclcl}
\nonterminal{xseq} & $::=$ & \nonterminal{x} & \quad & (singleton sequence)\\
  & \alt & & & (empty sequence)\\
  & \alt & \verb+(+$x_{1}$\verb+,+ $\syndots\!$\verb+,+ $x_{n}$\verb+)+ & & (sequence, $n\geq1$)
\end{longtable}
Here the ``$\,\syndots$'' used means syntactic iteration, which is distinct
from (and must not be confused with) ``\wildcardrow'' which is a reserved
keyword of the language.
\item Alternative forms for each phrase class are in order of decreasing
  precedence; this resolves ambiguity in parsing (or so the Definition
  assures us, guiding us to Appendix B)
\item L (resp., R) means left (resp., right) association
\item Syntax of types binds more tightly than the syntax of expressions
\item Each iterated construct (e.g., \nonterminal{match}, $\syndots$) extends as far
  right as possible; therefore, in theory, parentheses may be needed
  around an expression which terminates with a match (like ``\texttt{fn}
  \nonterminal{match}'') if this occurs within a larger match. (I guess you could
  say \emph{care must be taken if you play with matches}\dots)
\end{enumerate}
Unspoken rules:
\begin{enumerate}[resume]
\item ``nonterminals'' are italicized (\nonterminal{like so}), ``terminals''
  are written in teletype (\texttt{like so}).
\end{enumerate}
\end{convention}

\begin{comment}{``Match extends as far right as possible'' context-free?}
I am not sure if this criteria ``extends as far right as possible'' is
context-free, i.e., we might accidentally have a context-sensitive (or
worse) grammar accidentally with this seemingly innocuous condition. 

Scott, Johnstone, and Walsh~\cite{scott2023multiple} have written a
paper about this sort of condition. It seems like it is innocent enough,
but there may be difficulties or unintended complications with how the
Definition uses it.
\end{comment}

\begin{definition}{Syntactic classes for Core}
The syntactic classes for Core, which the Definition calls
\define{Core Phrase Classes}, consists of the following:
\makeatletter
\tabskip\@centering
\halign to\textwidth
{#\hfil\tabskip1em&#\hfil\tabskip\@centering\cr
AtExp   & atomic expressions \cr
ExpRow  & expression rows \cr
Exp     & expressions \cr
Match   & matches \cr
Mrule   & match rules \cr
\noalign{\vspace{2mm}}
%\cr
Dec     & declarations \cr
ValBind & value bindings \cr
TypBind & type bindings \cr
DatBind & datatype bindings \cr
ConBind & constructor bindings \cr
%version 1: Constrs & datatype constructions \cr
ExBind  & exception bindings \cr
\noalign{\vspace{2mm}}
%\cr
AtPat   & atomic patterns \cr
PatRow  & pattern rows \cr
Pat     & patterns \cr
\noalign{\vspace{2mm}}
%\cr
Ty      & type expressions \cr
TyRow   & type-expression rows \cr
}
\makeatother
\noindent As usual, the convention is to use italicized lowercase names \nonterminal{x} for
metavariables ranging over the syntactic class X.
\end{definition}

\begin{grammar}{Syntax for patterns}
The syntax for patterns:
\begin{longtable}{rcll}
\phantomsection\label{grammar:atpat}\nonterminal{atpat} & $::=$ & \wildcard & wildcard\\
& \alt & \nonterminal{scon} & special constant\\
& \alt & \optional{\texttt{op}} \textit{longvid} & value identifier\\
& \alt & \verb+{+ \optional{\textit{patrow}} \verb+}+ & record\\
& \alt & \verb+{+ \textit{pat} \verb+}+ &\\
\phantomsection\label{grammar:patrow}\nonterminal{patrow} & $::=$ & \wildcardrow & wildcard\\
& \alt & \nonterminal{lab} \verb+=+ \nonterminal{pat} \optional{\texttt{\char`,} \nonterminal{patrow}} & pattern row\\
\phantomsection\label{grammar:pat}\nonterminal{pat} & $::=$ & \nonterminal{atpat} & atomic\\
& \alt & \optional{\texttt{op}} \textit{longvid} \nonterminal{atpat} & constructed pattern\\
& \alt & $pat_{1}\;\;vid\;\;pat_{2}$ & infixed value construction\\
& \alt & \nonterminal{pat} \texttt{:} \nonterminal{ty} & typed \\
& \alt & \optional{\texttt{op}}\nonterminal{\vid}\optional{\texttt{:} \nonterminal{ty}} \texttt{as} \nonterminal{pat} & layered
\end{longtable}
\end{grammar}

\begin{grammar}{Syntax for types}
The syntax for types consists of the production rules:
\begin{longtable}{rcll}
\phantomsection\label{grammar:ty}\nonterminal{ty} & $::=$ & \nonterminal{tyvar} & type variable\\
& \alt & \texttt{\char`\{} \optional{\textit{tyrow}} \texttt{\char`\}} & record type expression\\
& \alt & \nonterminal{tyseq} \nonterminal{longtycon} & type construction\\
& \alt & \nonterminal{ty} \texttt{->} \nonterminal{ty\/$'$} & function type expression (R)\\
& \alt & \texttt{(} \nonterminal{ty} \texttt{)} & \\
\phantomsection\label{grammar:tyrow}\nonterminal{tyrow} & $::=$ & \nonterminal{lab} \texttt{:} \nonterminal{ty} \optional{\texttt{\char`,} \nonterminal{tyrow}} & type-expression row\\
\end{longtable}
\end{grammar}

\begin{grammar}{Expressions}
Remember that ``special constants'' is what the Definition refers to
``literal constants''.
\begin{longtable}{rcll}
\phantomsection\label{grammar:atexp}\nonterminal{atexp} & $::=$ & \nonterminal{scon} & special constant\\
& \alt& \optional{\texttt{op}} \textit{longvid} & value identifier\\
&\alt&\verb+{+ \optional{\textit{exprow}} \verb+}+ & record\\
&\alt&\texttt{let} \hyperref[grammar:dec]{\nonterminal{dec}} \texttt{in} \nonterminal{exp} \texttt{end} & local declaration\\
&\alt&\texttt{(} \nonterminal{exp} \texttt{)} & \\
\phantomsection\label{grammar:exprow}\nonterminal{exprow} & $::=$ & \nonterminal{lab} \texttt{=} \nonterminal{exp} \optional{\texttt{\char`\,} \nonterminal{exprow}}
& expression row\\
\phantomsection\label{grammar:exp}\nonterminal{exp} & $::=$ & \nonterminal{atexp} & atomic\\
&\alt& \nonterminal{exp} \nonterminal{atexp} & application (L)\\
&\alt& $exp_{1}\;\;vid\;\;exp_{2}$ & infixed application\\
&\alt& \nonterminal{exp} \texttt{:} \hyperref[grammar:ty]{\nonterminal{ty}} & typed (L)\\
&\alt& \nonterminal{exp} \texttt{handle} \hyperref[grammar:match]{\nonterminal{match}} & handle exception\\
&\alt&\texttt{raise} \nonterminal{exp} & raise exception\\
&\alt&\texttt{fn} \hyperref[grammar:match]{\nonterminal{match}} & function\\
\end{longtable}
\end{grammar}

\begin{grammar}{``Matches''}
In \texttt{case}-expressions, and function declarations (and
applications), we have pattern matching based on a set of rules. These
are usually called ``matches''.
\begin{longtable}{rcl}
\phantomsection\label{grammar:match}\nonterminal{match} & $::=$ & \nonterminal{mrule} \optional{\texttt{\char`\|} \nonterminal{match}} \\
\phantomsection\label{grammar:mrule}\nonterminal{mrule} & $::=$ & \hyperref[grammar:pat]{\nonterminal{pat}} \texttt{=>} \hyperref[grammar:exp]{\nonterminal{exp}}\\
\end{longtable}
\end{grammar}

\begin{grammar}{Core declarations}
This is the heart of what top-level programs look like, they're a
sequence of declarations (of some sort).
\begin{longtable}{rcll}
\phantomsection\label{grammar:dec}\nonterminal{dec} & $::=$ & \texttt{val}
\nonterminal{tyvarseq} \nonterminal{valbind} & value declaration\\
&\alt&\texttt{type} \nonterminal{typbind} & type declaration\\
&\alt&\texttt{datatype} \nonterminal{datbind} & datatype declaration\\
&\alt&\texttt{datatype} \nonterminal{tycon} \texttt{-=-} \texttt{datatype} \nonterminal{longtycon}
& datatype replication\\
&\alt&\texttt{abstype} \nonterminal{datbind} \texttt{with} \nonterminal{dec} \texttt{end} &
abstype declaration\\
&\alt&\texttt{exception} \nonterminal{exbind} & exception declaration\\
&\alt&\texttt{local} $dec_{1}$ \texttt{in} $dec_{2}$ \texttt{end} &
local declaration\\
&\alt&\texttt{open} $longstrid_{1}$ $\syndots$ $longstrid_{n}$ & open
declaration ($n\geq1$)\\
&\alt&&empty declaration\\
&\alt&$dec_{1}$ \optional{\texttt{;}} $dec_{2}$ & sequential declarations\\
&\alt&\texttt{infix} \optional{\textit{d}} $vid_{1}$ $\syndots$ $vid_{n}$
&infix (L) directive\\
&\alt&\texttt{infixr} \optional{\textit{d}} $vid_{1}$ $\syndots$ $vid_{n}$
&infix (R) directive\\
&\alt&\texttt{nonfix} \optional{\textit{d}} $vid_{1}$ $\syndots$ $vid_{n}$
&nonfix directive\\
%\phantomsection\label{grammar:}
%% \end{longtable}
%% \begin{longtable}{rcl}
\phantomsection\label{grammar:valbind}\nonterminal{valbind} & $::=$ & \hyperref[grammar:pat]{\textit{pat}}
\texttt{=} \hyperref[grammar:exp]{\textit{exp}} \optional{\texttt{and} \textit{valbind}}\\
&\alt&\texttt{rec} \textit{valbind}\\
\phantomsection\label{grammar:typbind}\nonterminal{typbind} & $::=$ &
\textit{tyvarseq tycon} \texttt{=} \hyperref[grammar:ty]{\textit{ty}} \optional{\texttt{and} \textit{typbind}}\\
\phantomsection\label{grammar:datbind}\nonterminal{datbind} & $::=$ &
\textit{tyvarseq tycon} \texttt{=} \textit{conbind} \optional{\texttt{and} \textit{datbind}}\\
\phantomsection\label{grammar:conbind}\nonterminal{conbind} & $::=$ &
\optional{\texttt{op}}\textit{vid} \optional{\texttt{of} \hyperref[grammar:ty]{\textit{ty}}} \optional{\texttt{\char`\|} \textit{datbind}}\\
\phantomsection\label{grammar:exbind}\nonterminal{exbind} & $::=$ &
\optional{\texttt{op}}\textit{vid} \optional{\texttt{of} \hyperref[grammar:ty]{\textit{ty}}} \optional{\texttt{and} \textit{exbind}}\\ 
&\alt&\optional{\texttt{op}}\textit{vid} \texttt{=} \optional{\texttt{op}} \textit{longvid} \optional{\texttt{and} \textit{exbind}}\\ 
\end{longtable}
\end{grammar}


\begin{remark}{Ambiguous grammar for \nonterminal{dec}}
Rossberg~\cite{rossbert2018defects} observes that the grammar given for
\nonterminal{dec} is highly ambiguous. For example, combining the empty
declaration \emph{and} sequencing allows the derivation of arbitrary
sequences of empty declarations for \emph{any} input.

A second ambiguity Rossberg points out is that a sequence of the form
``\nonterminal{dec$_{1}$} \nonterminal{dec$_{2}$} \nonterminal{dec$_{3}$}''
can be reduced in two ways to \nonterminal{dec}: either via
``\nonterminal{dec$_{12}$} \nonterminal{dec$_{3}$}'' or via ``\nonterminal{dec$_{1}$} \nonterminal{dec$_{23}$}''.
Rossberg cites Kahrs~\cite[\S8.3]{kahrs1993mistakes} for this second
ambiguity, which Kahrs expands upon with several more severe ambiguities.
\end{remark}

\section{Syntactic restrictions}

\begin{clause}{No repeated labels in records}
No expression row, pattern row, or type-expression row may bind the same
\nonterminal{lab} twice.
\end{clause}

\begin{clause}{Cannot bind same identifier twice}
No binding \nonterminal{valbind}, \nonterminal{typbind}, \nonterminal{datbind}, or \nonterminal{exbind} may bind the
same identifier twice. This applies also to value identifiers within a
\nonterminal{datbind}. 
\end{clause}

\begin{clause}{Type variables sequence uniqueness}
No \nonterminal{tyvarseq} may contain the same \nonterminal{tyvar} twice. So \verb|('a, 'b, 'a)|
is illegal since \verb|'a| appears twice.
\end{clause}

\begin{clause}{\texttt{rec} restriction for derived function form}
For each value binding ``\nonterminal{pat} \texttt{=} \nonterminal{exp}'' within \texttt{rec},
\nonterminal{exp} must be of the form ``\texttt{fn} \nonterminal{match}''. The derived form of
function-value binding given in Appendix A necessarily obeys this restriction.
\end{clause}

\begin{clause}{No binding primitive data constructors}
No \nonterminal{datbind}, \nonterminal{valbind}, or \nonterminal{exbind} may bind \texttt{true},
\texttt{false}, \texttt{nil}, \texttt{::}, or \texttt{ref}.
\end{clause}

\begin{clause}{No data or exception binding to \texttt{it}}
No \nonterminal{datbind} or \nonterminal{exbind} may bind \texttt{it}.
\end{clause}

\begin{clause}{Real constants never appear in patterns}
No real constant may occur in a pattern. (Really, unless you're a
numerical analyst, you should be staying away from floating-point
arithmetic altogether\dots)
\end{clause}

\begin{remark}{Problem stems from \texttt{real} is not equality type}
The difficulty with the previous clause (real constants cannot appear in
patterns) stems from \texttt{real} is not an equality type. This (real
not being an equality type) has felt wrong to me, since the IEEE 754
standard \emph{does} define equality for floating-point
numbers.

Moreover, it's the \emph{only} type which is not an equality
type, and that causes serious complications --- it's the entire reason we
need to distinguish ``equality types'' from ``non-equality types''! But
if we allow equality for real constants, the only possible surprise is
that \texttt{NaN} is never equal to anything else (i.e., we lose
reflexivity). 
\end{remark}

\begin{clause}{Type variable restrictions in value declarations}
In a value declaration ``\texttt{val} \nonterminal{tyvarseq} \nonterminal{valbind}'', if
\nonterminal{valbind} contains another value declaration ``\texttt{val} \nonterminal{tyvarseq\/$'$}
\nonterminal{valbind\/$'$}'', then \nonterminal{tyvarseq} and \nonterminal{tyvarseq\/$'$} must be disjoint.

In other words, no type variable may be scoped by two value declarations
of which one occurs inside the other.

This restriction applies after \nonterminal{tyvarseq} and \nonterminal{tyvarseq\/$'$} have been
extended to include implicitly scoped type variables (as explained in
Section 4.6).
\end{clause}