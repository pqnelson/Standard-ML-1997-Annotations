\chapter{Preface}

\begin{remark}{Work in progress}
This is a perpetual work in progress. Originally I hoped to learn enough
of the Definition of Standard ML to recast it in a more palatable form,
but that seems now to be a fool's errand.

I suspect I will work on this for a while, get bored, abandon it, then
revisit it much later on and iterate this process.

\begin{quote}
For small erections may be finished by their first architects; grand
ones, true ones, ever leave the copestone to posterity. God keep me from
ever completing anything. This whole book is but a draught---nay, but
the draught of a draught. Oh, Time, Strength, Cash, and Patience!

--- Herman Melville, \textit{Moby Dick}, Chapter 32 (1851)
\end{quote}
\end{remark}

\begin{comment}{Commentary or reconstruction?}
This is an attempt to understand the 1997 Revised
Definition~\cite{milner1997definition}, but I am not sure if it should
be a ``rational reconstruction'' (using anachronistic tools and terms to
``build from scratch'' the contents of the definition) or a
``commentary'' (trying to articulate the rather terse text).

Milner and Tofte wrote a commentary~\cite{milner1991commentary} on the
original 1990 Definition~\cite{milner1990definition}.  It's arguable how
similar the two Definitions are to each other. This makes the usefulness
of Milner and Tofte's commentary subject to debate.

I want to be able to reason about Standard ML code with the results of
this document. That's my goal. Towards that end, I suppose I need a
``bit of both''.

\emph{Specifically, this document is geared towards \textbf{understanding}
the Definition.} So I guess that makes it a ``commentary'' after all\dots
\end{comment}

\begin{comment}{``The Definition'' refers to the 1997 definition}
We will stop citing the 1997 Revised
Definition~\cite{milner1997definition} and henceforth refer to it as
``the Definition'' (capitalized). If we want to refer to the 1990
Definition~\cite{milner1990definition}, then we will refer to it as
``the 1990 Definition'' (also capitalized).

When quoting explicitly from the Definition, we will still cite the
bibliography entry, as well as provide the pages found in the printed
version. 
\end{comment}

\begin{comment}{Organization of this document}
This commentary is organized ``in parallel'' to the Definition. That is
to say, the chapters from 2 onwards is commenting on their
correspondingly numbered chapters in the 1997 revised Definition.

This contrasts with Milner and Tofte's commentary, which works through
the 1990 Definition starting with evaluating expressions, then the static
typechecking. Their commentary is probably more enjoyable (and certainly
better written) than mine, but I am trying to understand the 1997
Definition completely.
\end{comment}

\begin{comment}{Method of working, species of propositions}
We will break up the 1997 Definition into atomic propositions,
definitions, and inference rules; then we will try to reconstruct what
we can from these components. Each atomic proposition is numbered within
each section, for ease of reference.

There are different ``species'' of propositions we offer. Right now, the
basic heuristics for their meaning:
\begin{enumerate}
\item ``Deviation'' means we are intentionally departing from the
  definition somehow (e.g., simplifying it by refusing to allow ``='' to
  be a valid identifier that could be defined);
\item ``Clauses'' are rules, constraints, conditions, etc., imposed or
  required by the Definition; (this term is chosen to be consistent with
  the terminology found in, e.g., IEEE standards documents --- it is not
  used by the Definition);
\item ``Rules'' are inference rules the Definition gives (plus some of my
  commentary explaining its signficance);
\item ``Comments'' are comments which I am producing for the sake of
  commenting, clarifying, explicating;
\item ``Remarks'' are\dots remarks --- they are asides, not necessarily
  shedding light on the meaning of a clause or ``ur text'', but
  comparisons to how other languages do things, or what we've learned
  since the Definition was drafted; you know, things we find\dots remarkable\dots;
\item ``Puzzle'' are open questions which, as far as I know, do not have
  a solution (and I would be very excited to learn more about);
\item ``Definition'' defines a new phrase or term;
\item ``Convention'' usually refers to conventions adopted for
  metavariables or notation;
\item ``Example'' offers an illustrative example of some concept,
  definition, or term;
\item ``Theorem'' is self-explanatory;
\item ``Proposition'' is usually a theorem which is not as important;
\end{enumerate}
\end{comment}

\begin{deviation}{Fragment of Standard ML}\label{deviation:no-redefinitions}
We will work with the ``fragment'' of Standard ML which has the
following constraints: in the same scope, we cannot redefine a function,
constant, whatever.

When we declare a new structure (or signature), we \emph{can} ``reuse''
an identifier and any references using that identifier refers to the
``new'' definition.

Any duplicate definitions --- or reusing the same identifier in the same
scope --- for the same structure, type, variable, or data constructor
will be flagged as an error.

This is the same constraint that, e.g., Haskell has on redefinitions.
\end{deviation}

\begin{puzzle}{Formalize the definition using \FS0/?}
Can we formalize the Definition using Feferman's \FS0/ set theory?
Ostensibly, this seems plausible, but I honestly do not know.

Arguably, the more natural metatheory would be some flavor of HOL.
\end{puzzle}

\begin{remark}{Acknowledgements}
The formatting of propositions is heavily inspired by Alan
U.\ Kennington's book on differential geometry~\cite{kennington2024dg}.

I am also heavily referring to the source of the Definition,
which has been made available online for free.\footnote{\url{https://github.com/SMLFamily/The-Definition-of-Standard-ML-Revised/}}
I have borrowed some of their formatting macros.
The wildcard pattern macro ``\wildcard'' I borrowed from LHS2\TeX\ (if
memory serves me), but the syntactic ellipses ``\syndots'' as well as
the wildcard row pattern ``\wildcardrow'' I have hacked together.

Many of the comments which Kahrs~\cite{kahrs1993mistakes,kahrs1996addenda}
and Rossberg~\cite{rossberg2018defects} have made have been integrated
into this text, with citations to who observed what.
\end{remark}