\chapter{Syntax of Modules}

\section{Reserved words}

\begin{definition}{Reserved words}\label{defn:syntax-modules:reserved-words}
The modules have additional reservewords, namely:
\begin{verbatim}
             eqtype    functor   include   sharing   sig
             signature   struct   structure   where   :>
\end{verbatim}
\end{definition}

\section{Identifiers}

\begin{definition}{Functor identifier syntactic class FunId}
The functor identifier syntactic class, denoted \FunId, consists of
alphanumeric identifiers not starting with a prime.
\end{definition}

\begin{definition}{Signature Identifier syntactic class SigId}
The signature identifier syntactic class, denoted \SigId, also consists
of alphanumeric identifiers not starting with a prime.
\end{definition}

\begin{convention}{Disjointness of identifier classes}
Henceforth, the Definition considers all identifier classes to be
disjoint and discernible by grammatical rules.
\end{convention}

\section{Infixed Operators}

\begin{clause}{Additional scoping rules}
In addition to the rules governing the scope of infixed operators in the
Core language~\zref{syntax-core:infixed-operator:scope}, there is a
further scope limitation: if \nonterminal{dir} occurs in a
structure-level declaration \nonterminal{strdec} in any of the phrases
\[ \texttt{let}\ \nonterminal{strdec}\ \texttt{in}\ \syndots\ \texttt{end} \]
\[ \texttt{local}\ \nonterminal{strdec}\ \texttt{in}\ \syndots\ \texttt{end} \]
\[ \texttt{struct}\ \nonterminal{strdec}\ \texttt{end} \]
then the scope of \nonterminal{dir} does not extend beyond the phrase.

An immediate consequence: fixity is local to a basic structure
expression (in particular, to such an expression occurring as a functor body).
\end{clause}

\section{Grammar for Modules}

\begin{definition}{Syntactic classes for modules}
The syntactic classes for modules, which the Definition calls
\define{Module Phrase Classes}, consists of the following:
\makeatletter{}
\tabskip\@centering
\halign to\textwidth
{#\hfil\tabskip1em&#\hfil\tabskip\@centering\cr
StrExp & structure expressions \cr
StrDec & structure-level declarations \cr
StrBind & structure bindings \cr
\cr
SigExp & signature expressions \cr
SigDec & signature declarations \cr
SigBind & signature bindings \cr
\cr
Spec & specifications \cr
ValDesc & value descriptions\cr
TypDesc & type descriptions\cr
DatDesc & datatype descriptions\cr
ConDesc & constructor descriptions\cr
ExDesc & exception descriptions\cr
StrDesc & structure descriptions\cr
%\ adhocinsertion{\thetypabbr}{2mm}{WhereBind & where bindings\cr}
%\adhocdeletion{\thetypabbr}{2mm}{SharEq & sharing equations\cr}
\cr
FunDec & functor declarations\cr
FunBind & functor bindings\cr%% \adhocdeletion{\thenofuncspec}{3mm}{FunSigExp & functor signature expressions\cr
%% FunSpec & functor specifications\cr
%% FunDesc & functor descriptions\cr}
TopDec  & top-level declarations\cr
}
\makeatother
\end{definition}

\section{Grammar for Modules}

\begin{grammar}{Structures}
  The syntax for structures is fairly straightforward. Note that
  the rule for structure declarations \nonterminal{strdec} includes the
  rule for core declarations \nonterminal{dec}, which is particularly
  important later on.
\begin{longtable}{rcll}
\phantomsection\label{grammar:strexp}\nonterminal{strexp} & $::=$ & \texttt{struct} \nonterminal{strdec} \texttt{end}
& basic\\
& \alt & \nonterminal{longstrid} & structure identifier\\
&\alt& \nonterminal{strexp} \texttt{:} \hyperref[grammar:sigexp]{\nonterminal{sigexp}} & transparent constraint\\
&\alt& \nonterminal{strexp} \texttt{:>} \hyperref[grammar:sigexp]{\nonterminal{sigexp}} & opaque constraint\\
&\alt& \nonterminal{funid}\texttt{(}\nonterminal{strexp}\texttt{)} & functor application\\
&\alt&\texttt{let} \nonterminal{strdec} \texttt{in} \nonterminal{strexp} \texttt{end} & local declaration\\
\phantomsection\label{grammar:strdec}\nonterminal{strdec} & $::=$ & \hyperref[grammar:dec]{\nonterminal{dec}}
& declaration\\
&\alt&\texttt{let} \texttt{structure} \nonterminal{strbind} & structure\\
&\alt&\texttt{local} \nonterminal{strdec}$_{1}$ \texttt{in} \nonterminal{strdec}$_{2}$ \texttt{end} & local declaration\\
&\alt& & empty\\
&\alt&\nonterminal{strdec}$_{1}$ \optional{\texttt{;}} \nonterminal{strdec}$_{2}$ & sequential\\
\phantomsection\label{grammar:strbind}\nonterminal{strbind} & $::=$ & \nonterminal{strid} \texttt{=} \nonterminal{strexp} \optional{\texttt{and} \nonterminal{strbind}} & \\
\end{longtable}
\end{grammar}

\begin{remark}{Ambiguous grammar for \nonterminal{strdec}}
Rossberg~\cite{rossberg2018defects} notes that the production rule for
sequential structure declarations ``\nonterminal{strdec}$_{1}$ \optional{\texttt{;}} \nonterminal{strdec}$_{2}$''
is ambiguous. Is it left-associative or right-associative?

I'd advocate for making it right recursive, so we could write a simple
recursive descent parser for Standard ML.
\end{remark}

\begin{grammar}{Signatures}
The syntax for signatures:
\begin{longtable}{rcll}
\phantomsection\label{grammar:sigexp}\nonterminal{sigexp} & $::=$ & \texttt{sig} \nonterminal{spec} \texttt{end} & basic\\
  &\alt& \nonterminal{sigid} & signature identifier\\
&\alt&\nonterminal{sigexp} \texttt{where type} \nonterminal{tyvarseq} \nonterminal{longtycon}
\texttt{=} \nonterminal{ty} & type realisation\\
\phantomsection\label{grammar:sigdec}\nonterminal{sigdec} & $::=$ & \texttt{signature} \nonterminal{sigbind} & \\
\phantomsection\label{grammar:sigbind}\nonterminal{sigbind} & $::=$
& \nonterminal{sigid} \texttt{=} \nonterminal{sigexp}
\optional{\texttt{and} \nonterminal{sigbind}} & \\
\end{longtable}
\end{grammar}

\begin{grammar}{Specifications}
The syntax for specifications which appear in the body of a signature:
\begin{longtable}{rcll}
%\phantomsection\label{grammar:}\nonterminal{} & $::=$ & &
\phantomsection\label{grammar:spec}\nonterminal{spec} & $::=$ &
\texttt{val} \nonterminal{valdesc} & value\\
&\alt&\texttt{type} \nonterminal{typdesc} & type\\
&\alt&\texttt{eqtype} \nonterminal{typdesc} & equality type\\
&\alt&\texttt{datatype} \nonterminal{datdesc} & datatype\\
&\alt&\texttt{datatype} \nonterminal{tycon} \texttt{-=-}
\texttt{datatype} \nonterminal{longtycon} & replication\\
&\alt& \texttt{exception} \nonterminal{exdesc} & exception\\
&\alt& \texttt{structure} \nonterminal{strdesc} & structure\\
&\alt& \texttt{include} \nonterminal{sigexp} & include\\
&\alt& & empty\\
&\alt& \nonterminal{spec}$_{1}$ \optional{\texttt{;}} \nonterminal{spec}$_{2}$ & sequential\\
&\alt&\nonterminal{spec} \texttt{sharing type}
\nonterminal{longtycon}$_{1}$ \texttt{=} $\syndots$ \texttt{=} \nonterminal{longtycon}$_{n}$
& sharing ($n\geq2$)\\
\phantomsection\label{grammar:valdesc}\nonterminal{valdesc} & $::=$ &
\nonterminal{vid} \texttt{:} \nonterminal{ty} \optional{\texttt{and} \nonterminal{valdesc}} & \\
\phantomsection\label{grammar:typdesc}\nonterminal{typdesc} & $::=$ &
\nonterminal{tyvarseq} \nonterminal{tycon} \optional{\texttt{and} \nonterminal{typdesc}} & \\
\phantomsection\label{grammar:datdesc}\nonterminal{datdesc} & $::=$ & \nonterminal{tyvarseq} \nonterminal{tycon} \texttt{=} \nonterminal{condesc} \optional{\texttt{and} \nonterminal{datdesc}} & \\
\phantomsection\label{grammar:condesc}\nonterminal{condesc} & $::=$ & \nonterminal{vid} \optional{\texttt{of} \nonterminal{ty}} \optional{\texttt{\char`\|} \nonterminal{condesc}} & \\
\phantomsection\label{grammar:exdesc}\nonterminal{exdesc} & $::=$ & \nonterminal{vid} \optional{\texttt{of} \nonterminal{ty}}
\optional{\texttt{\char`\|} \nonterminal{exdesc}} & \\
\phantomsection\label{grammar:strdesc}\nonterminal{strdesc} & $::=$ & \nonterminal{strid} \texttt{:} \nonterminal{sigexp} \optional{\texttt{and} \nonterminal{strdesc}} & \\
\end{longtable}
\end{grammar}

\begin{grammar}{Functor declarations}
The syntax for declaring a functor reuses the syntax for structure
expressions. Also observe that it is impossible to have a functor with
$0$ arguments, or with $n>1$ arguments.
\begin{longtable}{rcll}
%\phantomsection\label{grammar:}\nonterminal{} & $::=$ & & \\
\phantomsection\label{grammar:fundec}\nonterminal{fund} & $::=$ & \texttt{functor} \nonterminal{funbind} & \\
\phantomsection\label{grammar:funbind}\nonterminal{funbind} & $::=$ &
\nonterminal{funid} \texttt{(}\nonterminal{strid} \texttt{:} \nonterminal{sigexp}\texttt{)} \texttt{=} \hyperref[grammar:strexp]{\nonterminal{strexp}}
\optional{\texttt{and} \nonterminal{funbind}}
& functor binding\\
\end{longtable}
\end{grammar}

\begin{grammar}{Top-level declarations}
The syntax for top-level declarations (which forms the basis of a
program) consists of a sequence of one or more structure declarations
[remember: core-language declarations are swept into structure declarations],
signature declarations, and/or functor declarations:
\begin{longtable}{rcll}
%\phantomsection\label{grammar:}\nonterminal{} & $::=$ & & \\
  \phantomsection\label{grammar:topdec}\nonterminal{topdec} & $::=$ &
\hyperref[grammar:strdec]{\nonterminal{strdec}} \optional{\nonterminal{topdec}} & structure-level declaration \\
&\alt& \hyperref[grammar:sigdec]{\nonterminal{sigdec}} \optional{\nonterminal{topdec}} & signature declaration \\
&\alt& \hyperref[grammar:fundec]{\nonterminal{fundec}} \optional{\nonterminal{topdec}} & functor declaration \\
\end{longtable}
\end{grammar}

\begin{puzzle}{LL(1) grammar for Standard ML?}
Can we revise the grammar for Standard ML (both Core and Modules) to
make it explicitly LL(1)?
\end{puzzle}

\section{Syntactic Restrictions}

\begin{clause}{No rebinding the same identifier in simultaneous declarations}
No binding \nonterminal{strbind}, \nonterminal{sigbind}, or
\nonterminal{funbind} may bind the same identifier twice.
The Definition means by this, if we have something like 
``\texttt{functor foo(arg1 : sig1) = ... and foo(arg2 : sig2) = ...}''
then we have both functor declarations use the same identifier
``\texttt{foo}''.

However, our deviation~\zref{deviation:no-redefinitions} already imposes
a stronger constraint.
\end{clause}

\begin{clause}{Specifications cannot reuse identifiers}
No description \nonterminal{valdesc}, \nonterminal{typdesc},
\nonterminal{datdesc}, \nonterminal{exdesc}, or \nonterminal{strdesc}
may describe the same identifier twice. This also applies to value
identifiers within a \nonterminal{datdesc}. (Again, we're imposing a
stronger constraint\dots)
\end{clause}

\begin{clause}{Type variables must be distinct}
No \nonterminal{tyvarseq} may contain the same \nonterminal{tyvar} twice.
\end{clause}

\begin{clause}{Type variables in data constructor appear in type constructor}
Any \nonterminal{tyvar} appearing on the right side of a
\nonterminal{datdesc} of the form ``\nonterminal{tyvarseq}
\nonterminal{tycon} \texttt{=} $\syndots$'' must occur in the
\nonterminal{tyvarseq}. Similarly, in signature expressions of the form
``\nonterminal{sigexp} \texttt{where type} \nonterminal{tyvarseq}
\nonterminal{longtycon} \texttt{=} \nonterminal{ty}'', any
\nonterminal{tyvar} occurring in \nonterminal{ty} must occur in \nonterminal{tyvarseq}. 
\end{clause}

\begin{clause}{Specification identifiers are not reserved words}
No \nonterminal{datdesc}, \nonterminal{valdesc}, or \nonterminal{exdesc}
``may describe'' [whatever that means] \texttt{true}, \texttt{false},
\texttt{nil}, \texttt{::}, or \texttt{ref}.

Further, no \nonterminal{datdesc} or \nonterminal{exdesc} ``may describe''
\texttt{it}. 
\end{clause}

\begin{comment}{Constraining our deviation}
So our deviation~\zref{deviation:no-redefinitions} from the ``full''
language imposed the constraint that redefinitions are illegal, but we
can ``shadow'' an identifier if it is inside a structure. We see
that the Definition further limits the possible identifiers we can
shadow \emph{in general}.
\end{comment}

\begin{clause}{Top declaration restriction}\label{clause:syntax-modules:top-declaration-restriction}
No \nonterminal{topdec} may contain, as an initial segment, a
\nonterminal{strdec} followed by a semicolon.
(Rossberg~\cite{rossberg2018defects} observes that the \emph{intention}
is to make parsing of top-level semicolons unambiguous, so that they
always terminate a program.)
\end{clause}

\begin{comment}{Semicolons treated worse than a dead dog}
It seems to me that semicolons are being treated worse than a dead dog,
and it might actually improve the language if every statement and
declaration had to end with a semicolons. Or if semicolons were banned
altogether. It's got to be one or the other, because right now it's this
muddied middle-of-the-road strategy (which is a great way to get hit by
a car).
\end{comment}

\begin{comment}{References on Standard ML syntax}
Scott and Jonstone~\cite{scott2023analysing} have re-examined Standard
ML's syntax using modern lexical tools.
\end{comment}