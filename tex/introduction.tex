\chapter{Introduction}

\begin{comment}{Basic judgements in the Definition}
  In the preface to the 1997
  Definition~\cite[pp.xi-xii]{milner1997definition}, the authors explain
  in the preface: 
  \begin{quotation}
    We shall now explain the keystone of our semantic method. First, we
    need a slight but important refinement. A phrase $P$ is never evaluated
    \textit{in vacuo} to a meaning $M$, but always against a background;
    this background --- call it $B$ --- is itself a semantic object,
    being a distillation of the meanings preserved from evaluation of
    earlier phrases (typically variable declarations, procedure
    declarations, etc.). In fact evaluation is background-dependent ---
    $M$ depends upon $B$ as well as upon $P$.

    The keystone of the method, then, is a certain kind of assertion about
    evaluation; it takes the form
    \begin{equation*}
B\vdash P\Rightarrow M
    \end{equation*}
    and may be pronounced: ``Against the background $B$, the phrase $P$
    evaluates to the meaning $M$''. \emph{The formal purpose of this
    Definition is no more, and no less, than to decree exactly which
    assertions of this form are true}. This could be achieved in many
    ways. We have chosen to do it in a structured way, as others have,
    by giving rules which allow assertions about a \emph{compound} phrase $P$
    to be inferred from assertions about its \emph{constituent} phrases
    $P_{1}$, \dots, $P_{n}$.
  \end{quotation}
  Unfortunately, programming language semantics evolved beyond this, and
  this approach seems archaic.
\end{comment}

\begin{comment}{Phases to the definition}
Again, we find in the first chapter of the 1997 Definition, the authors
explain (emphasis theirs):
\begin{quotation}
ML is an interactive language, and a \emph{program} consists of a
sequence of \emph{top-level declarations}; the execution of each
declaration modifies the top-level environment, which we call a
\emph{basis}, and reports the modification to the user.

In the execution of a declaration there are three phases:
\emph{parsing}, \emph{elaboration}, and \emph{evaluation}. Parsing
determines the grammatical form of a declaration. Elaboration, the
\emph{static} phase, determines whether it is well-typed and well-formed
in other ways, and records relevant type or form information in the
basis. Finally evaluation, the \emph{dynamic} phase, determines the
value of the declaration and records relevant value information in the
basis. Corresponding to these phases, our formal definition divides
into three parts: grammatical rules, elaboration rules, and evaluation
rules. Furthermore, the basis is divided into the \emph{static} basis
and the \emph{dynamic} basis; for example, a variable which has been
declared is associated with a type in the static basis and with a value
in the dynamic basis.
\end{quotation}
This gives us a clean way to order our discussion into three phases
(parsing, elaboration, and evaluation) and two sublanguages (Core and
Module). Since the Definition itself is structured in this manner, we
could use that fact to mirror its structure in our text.
\end{comment}